\label{sec:Kurzfasssung}
Im Kontext der Klassifizierung von Aktion und Objekten, soll die Optimierung von Hyperparametern bei künstlichen neuronalen Netzen mit Hilfe von Genetischen Algorithmen im Rahmen dieser Arbeit durchgeführt werden. Da es für die Hyperparameter keine klaren Regeln gibt und sie je nach Daten und Modell-Architektur anders gewählt werden müssen. Somit ist es für den Entwickler sehr zeitaufwendig, diese Hyperparameter von Hand auszuwählen. Daher ist das Ziel der Arbeit, ein Framework zu entwickeln und zu implementieren, in dem künstliche neuronale Netze automatisiert trainiert, getestet, Hyperparameter angepasst und ausgewertet werden. Dazu wurde ein Konzept entwickelt, in welchem die Hyperparameter als Gene des Genetischen Algorithmus umgesetzt werden. So werden die Hyperparameter nach dem Vorbild der Evolution über Generationen intelligent angepasst. Des Weiteren wurde der State-of-the-Art Algorithmus ausgesucht und implementiert, um den Genetischen Algorithmus vergleichen zu können. Es handelt sich um die Zufallssuche, welche zufällige Individuen im Suchraum testet. Beide Optimierungsalgorithmen wurden zur Optimierung von Hyperparametern in Fully Connected Networks und Convolutional Neural Networks verwendet. Zudem wurden verschiedene Datensätze verwendet, um eine unabhängige Aussage zu erreichen. Der Genetische Algorithmus konnte leider keine Verbesserungen im Vergleich zur Zufallssuche erreichen. Beide Algorithmen finden Hyperparameter, welche gleich gute Klassifizierungsergebnisse liefern. Sie unterscheiden sich nur im Nachkommabereich und sind somit vernachlässigbar klein. Dennoch hat der Genetische Algorithmus einen Vorteil: Über seine letzte Population kann eine zusätzliche Aussage getroffen werden, in welchem Bereich sich die Hyperparameter befinden müssen, um gute Ergebnisse zu erreichen. Dies ist mit der Zufallssuche nicht möglich, da diese nicht genügend gute Individuen findet, um diese Aussage zu treffen. Die wenigen die die Zufallssuche findet, sind ausreichend gut und können von dem Genetischen Algorithmus nicht übertroffen werden.