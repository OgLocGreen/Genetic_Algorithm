%% LaTeX template for FZI designed Document
%% by Matthias Huber and Florian Kuhnt
%%
%% version 0.4
%%
%% Problems, bugs and comments to
%% huber@fzi.de

\documentclass[parskip=true]{scrartcl}

\usepackage[utf8]{inputenc}
% Toggle the following two lines to switch between english and german layout
%\usepackage[english]{babel} % English
\usepackage[ngerman]{babel} % Neue deutsche Rechtschreibung und Silbentrennung.
\usepackage[T1]{fontenc}
%\usepackage{templates/theme}

\usepackage{lipsum, todonotes}	
\usepackage[nolist,nohyperlinks]{acronym}
\usepackage{caption}
\usepackage{float}


\newboolean{titel} 
\setboolean{titel}{true}
\newboolean{inhalt} 
\setboolean{inhalt}{true}
\newboolean{headfoot} 
\setboolean{headfoot}{true}

% Dokumentangaben
\newcommand{\dokumenttitel}{[GA]}
\newcommand{\untertitel}{[Untertitel]}
\newcommand{\version}{0.1}
\newcommand{\veroeffentlichung}{1.1.1970}
\newcommand{\autor}{[Chrisitian Heinzmann]}
\newcommand{\validfrom}{1.1.1970}
\newcommand{\state}{-vertraulich-}

%Titelseite?
%\setboolean{titel}{false} 
%Inhaltsverzeichnis?
%\setboolean{inhalt}{false}
%FZI Kopf- und Fußzeile?
%\setboolean{headfoot}{false}

%Inhaltsverzeichnis?
%\setboolean{inhalt}{false}
%FZI Kopf- und Fußzeile?
%\setboolean{headfoot}{false}

\begin{document}

%Seiten ohne Kopf- und Fußzeile sowie Seitenzahl
\pagestyle{empty}

\begin{center}
\begin{tabular}{p{\textwidth}}


\begin{center}
\includegraphics[scale=0.4]{img/Hska_logo.png}
\end{center}


\\

\begin{center}
\LARGE{\textsc{
Genetische Algorithmen zur Optimierung von Hyperparametern eines k"unstlichen neuronalen Netzes\\
}}
\end{center}

\\


\begin{center}
\large{Fakult"at f"ur Maschinenbau und Mechatronik \\
der Hochschule Karlsruhe \\
Technik und Wirtschaft \\}
\end{center}

\\

\begin{center}
\textbf{\Large{Bachelorarbeit}}
\end{center}


\begin{center}
vom 01.03.2018 bis zum 31.08.2018 \\
vorgelegt von
\end{center}

\begin{center}
\large{\textbf{Christian Heinzmann}} \\
\small{geboren am 18.02.1995 in Heilbronn} \\
\small{Martrikelnummer: 52550}
\end{center}

\begin{center}
\large{Winter Semester 2019}
\end{center}

\\

\\

\begin{center}
\begin{tabular}{lll}
\textbf{Professor} & & Prof. Dr.-Ing. habil. Burghart\\
\textbf{Co-Professor} & & Prof. Dr.-Ing. Olawsky\\
\textbf{Betreuer FZI:} & & M. Sc. Kohout\\
\end{tabular}
\end{center}

\end{tabular}
\end{center}

\addsec{Eigenständigkeitserklärung und Hinweis auf verwendete
Hilfsmittel}
\label{erklaerung}

Eigenständigkeitserklärung und Hinweis auf verwendete
Hilfsmittel.
Hiermit bestätige ich, dass ich den vorliegenden Praxissemesterbericht selbständig
verfasst und keine anderen als die angegebenen Hilfsmittel benutzt habe. Die Stellen
des Berichts, die dem Wortlaut oder dem Sinn nach anderen Werken entnommen sind,
wurden unter Angabe der Quelle kenntlich gemacht.\\
\\[1.5cm]
Datum:	\hrulefill\enspace Unterschrift: \hrulefill
\\[3.5cm]





\begin{figure}
\addsec{Ausschreibung}
\label{sec:Kurzfasssung}
\includegraphics[height=\dimexpr\textheight-0\baselineskip\relax,page=1]{img/Ausschreibung}
\label{fig:Praktikumsbeschreibung}
\end{figure}



% Beendet eine Seite und erzwingt auf den nachfolgenden Seiten die Ausgabe aller Gleitobjekte (z.B. Abbildungen), die bislang definiert, aber noch nicht ausgegeben wurden. Dieser Befehl fügt, falls nötig, eine leere Seite ein, sodaß die nächste Seite nach den Gleitobjekten eine ungerade Seitennummer hat. 
\cleardoubleoddpage

% pagestyle für gesamtes Dokument aktivieren
%\pagestyle{fancy}

%Inhaltsverzeichnis
\tableofcontents

%Verzeichnis aller Bilder
\listoffigures

%Verzeichnis aller Abkürzungen
\addsec*{Abkürzungsverzeichnis}
\begin{acronym}[Bash]
 \acro{KNN}[KNN]{Künstliches Neuronales Netz}
 \acro{test} {why no acro}
 \acro{Conv}{Convolutional}
 \acro{ESS}{Embedded Systems and Sensors Engineering}
 \acro{FZI}{Forschungszentrum Informatik}
 \acro{CNN}{Convolutional Neuronal Network}
 \acro{GA}{Genetic Algorithmen}
\end{acronym}



\section{Motivation}
\label{sec:Motivation}
In den letzten Jahren zeigte sich ein enormer Zuwachs an Publikationen und Forschungen zum Thema "Maschinelles Lernen". Die künstlichen neuronalen Netze dominieren dieses Feld, aber ihr Training und ihr Erfolg hängt noch immer von empirisch ausgesuchten Anfangsparametern, den sogenannten Hyperparametern, ab. Zu diesen Hyperparametern können Modell-Architektur, Verlustfunktion, Optimierung Algorithmen, Batchsize, Dropout und Lernrate gezählt werden. Momentan werden diese Hyperparameter meist nach Ermessen des Entwicklers ausgewählt, denn es gibt keine formalen Regeln. Dieses Auswählen und Testen der Hyperparameter beansprucht sehr viel Zeit und Mühe, da die anfängliche Auswahl der Parameter meist suboptimal ist und diese noch einige Male angepasst und getestet werden müssen. Dieses Optimieren der Hyperparameter sollte jedoch nicht zu den Aufgabe des Entwicklers gehören und sollte von Maschinen übernommen werden, welche in Bereich der Optimierung wesentlich effizienter arbeiten. \cite[p.~337]{francois2017deep}

\section{Aufgabenstellung}
\label{sec:Aufgabenstellung}
Ziel dieser Arbeit ist es die Optimierung von Hyperparametern für künstliche neuronale Netze zu automatisieren. Diese Optimierung soll Mithilfe des Genetischen Algorithmus erfolgen. Dafür muss als erstes ein automatisierter Trainings- und Auswerte-Prozess der künstlichen neuronalen Netze erstellt werden. Anschließend soll ein Konzept zur Optimierung von Hyperparametern mit Hilfe des Genetischen Algorithmus entwickelt und implementiert werde, wozu der automatische Trainings- und Auswerte-Prozess verwendet werden soll. Die Ergebnisse des Genetischen Algorithmus sollen anschließend mit den Ergebnissen des State-of-the-Art Algorithmus zur Optimierung von Hyperparametern gegenübergestellt und ausgewertet werden. Darüber hinaus soll eine Optimierung der Hyperparameter unter Betrachtung geringer Datenmengen durchgeführt werden. 

\section{Aufbau der Arbeit}
\label{sec:Aufbau der Arbeit}
Im zweiten Kapitel wird auf die Grundlagen der künstlichen neuronalen Netze eingegangen. Anschließend werden die zu optimierenden Hyperparameter definiert. Nachfolgend werden die Optimierungsalgortihmen erläutert, dazu gehören der Genetische Algorithmus mit seinen 5 Schritten und die Rastersuche bzw. die Zufallssuche.

Im dritten Kapitel wird auf den Stand der Technik eingegangen. Dazu werden die neusten Forschungsergebnisse erläutert. Zu diesen gehört das Populations Basierte Training(PBT) von Google und die NeuroEvolution of Augmenting Topologies(NEAT). Zudem werden Anwendungsbeispiele des Genetischen Algorithmus, welche in der Forschung als auch in der Industrie verwendetet werden aufgezeigt. 

Im vierten Kapitel wird auf das Konzept zur Optimierung der Hyperparameter eingegangen. Hierfür wird ein Konzept zum Genetischen Algorithmus zur Optimierung der Hyperparameter eines künstlichen neuronalen Netzes vorgestellt. Anschließend wird auf das Konzept zur Evaluierung und Auswertung mit Hilfe des Dichte-Diagramms eingegangen. 

Im fünften Kapitel wird der Systemaufbau erklärt. Es wird auf die verwendete Hardware eingegangen. Anschließend wird die exakte Implementierung des Genetischen Algorithmus, der Zufallssuche und der Evaluation sowie die Auswertung beschrieben.

Im sechsten Kapitel werden die Ergebnisse der Optimierung dargestellt und besprochen. Zudem wird die visuelle Auswertung zu jedem Hyperparameter aufgezeigt und erläutert. 

Im siebten und letzten Kapitel werden alle vorherigen Kapitel als Fazit zusammengefasst. Der Ausblick zum Thema bildet den Abschluss der Arbeit. 





\section{Grundlagen}
\label{sec:Grundlagen}


\subsection{Optimierungsgrundlagen}
Angenommen es soll ein Neuronales Netz mit k Layern und jNeuronen zur Klassifizierung von einfachen Handgeschrien-Zahlen erstellt werden. Der Entwickler entscheidet sich für ein 3 Layern Netz mit jeweils 3 Neuronen. Nach dem Training hat es die Genauigkeit von 85 Prozent. Ist dies Akzeptabel? Kann man sagen das für k 3 bzw. j 3 die optimale Lösung ist? Um dies zu beurteilen müssen viele Experimente durchgeführt werden. Die Frage ist, wie man den die Besten werte für k und j findet, der die Klassifizierung maximiert. Dies wird als Hyperparameter Optimierung bezeichnet. 

\subsection{Genetische Algorithmen}
Genetische Algorithmen sind heuristische Suchansätzem, die auf einer breitenbasis von Optimierungsproblemen angewendet werden können. Diese Flexibiliät macht sie für die Praxis für viele sehr attraktiv. Die Evulution ist Grundlage des Genetischen Aglorithmuses. durch die aktuelle Vielfalt und der Erfolg der Arten ist dies schon alleine ein guter Grund sich diesen Optimierungs Algortihmus näher anzuschauen. Denn sdiese Arten sind in der lage sich an ihre Umgebung anzupassen und sich zu zu komplexen Strukturen zuentwickelen, und somit das überleben in verschiedesten Umgebungen eröglichen. Hierbei ist die Paarung und Entwicklung von Nachkommen eine der Hauptprinipen des Erfolges der Evolution. In diesem Kapitel werden wir die Grundlage der GEnetischen Algorithmen näher anzuschauen. Beginnen wir mit der grundlage das es sich bei den Genetischen Algorithmen um einen Polulations ansatz handelt. Anschließend wird auf die wichtigesten genetischen Operatioren vorstellen darunter gehöhren, Selektion, Crossover und Muttation

Seite - 11 Genetic Algorithm Essentials


\iffalse
Algorithmus 1 zeigt den Pseudocode des grundlegenden genetischen Algorithmus, der Folgendes kann
dienen als Blaupause für viele verwandte Ansätze. Am Anfang eine Reihe von Lösungen,die als Population bezeichnet wird, wird initialisiert. Diese Initialisierung wird empfohlen.um zufällig den gesamten Lösungsraum abzudecken oder um Experten zu modellieren und einzubinden.
Wissen. Die Darstellung bestimmt den Initialisierungsprozess. Für BitfolgeDarstellungen ist eine zufällige Kombination von Nullen und Einsen sinnvoll, z.B.das anfängliche Zufallschromosom 1001001001001 als typische Bitfolge der Länge 10. Der Hauptgenerationskreislauf des Genetischen Algorithmus erzeugt neue Nachkommen.Kandidatenlösungen mit Crossover und Mutation, bis die Bevölkerung vollständig ist.

Algorithm 1 Basic Genetic Algorithm
1: initialize population
2: repeat
3: 		repeat
4:			fitness computation
5:			crossover
6:			mutation
7:			phenotype mapping
8:		until population complete
9:		selection of parental population
10: until termination condition

Seite - 11 Genetic Algorithm Essentials
\fi 

\subsubsection{Initzialisierung der Population}
Der Klassiche Genetische Algorithmus bassiert auf einer Reihe von Kandidatenlösungen. Dee größe der Populaton ist auch die Anzahl der Lösungen. Jede Lösung kann als einzelnes Induvidum gesehen werden und wird durch ein Chromsone representiert. Es gibt verschiedene möglichkeiten diese Gene dazustellen, wie z.B binär oder Dezimal.
Figure \ref{fig:chromosome} veranschaulicht ein Beispiel, wie eine Population aus vier Individuen(chromosomes)mit je einem Chromosom. Ein Chromso, besitzt wiederum  vier gene. Jedes dieser Gene ist durch eine binären zahl repräsentiert. 

\noindent%
\begin{figure}[H]
  \centering  
  \includegraphics[scale=0.3]{img/Chromsome-s134-PracticalComputerVion.png}
  \caption{Beispiel einer Polulation mit 4 induviduen (Chromsomen) welche vier binäre Gene besitzen \cite{rashid2017neuronale} }
  \label{fig:chromosome}
\end{figure}

Diese anfangs Poulation wird meißt Zufällig inizialisiert. Hier durch ist es möglich die größte abdeckung des Suchsraums zu gewähren. Dadurch besitzt die erste Generation eine sehr geringe Fittness dies verbessert sich aber im Lauf des Trainings. Durch Selection werde die nicht unnötigen/Contra-produktiven Individumen aussotiert. Bevor dies passiert muss aber erst die Bewertung durchgeführt werden.


\subsubsection{Bewertung aka Grade / Fittnesfunktion}
Nun besteht die erste Generation(Generation 0) aus einer Population mit völlig zufälligen Induviduen. Diese werden anhand einer für sine anwendung speizellen Fitnessfunktion bewerte. Dabei werden nicht einzelnen Gene bewertet sondern das ganze Genom/Chromoson/Idividum. Es wird also nicht berücksichtigt welches Gene sich positiv bzw. negativ auswirklen. Als Rückgabewert gibt die Fittnesfunktion uns einen Fittneswert, dabei steht ein höherer Fittnesswert stehts für eine höher Qualität an Individum. Da nun alle Individuen der Population bewertet wurden kann eine neue Generation erstellt werden.


\subsubsection{Weiterentwickeln aka Evolve}
Dazu werden aus zwei Elternpaaren ein neues Kind erstellt. Um die Elternpaare auszusuchen gibt es verschiedene Optionen.Da nun die Eltern fest stehen, wird per Crossover aus den beiden Elternpaaren oder aus dem Elternpool ein neues Kind generiert. Um bei den Genen eine höhere diversität zu gelangen werden die Kindergene noch mit einer Mutation versehen. Somit kann man einen höheren Suchraum(Abdeckungsgrad) abdecken. Nach dem eine neue Kind generation erstellt wurde wird der ganze vorgang so lange wiederhollt bis die geforderte Fintess ereicht wurde.

\paragraph{Auswahl der Elternpaare aka Select Parents}
Um eine convergenz richtung Optimalem Maximum bzw. Minmum zu schaffen werden die besten Elternteile der geraden bewertenen Generation ausgewählt.
Für die Auswahl gibt es verschiedene Ansätze, die bedeutesten werden genannt und erläutert.
\begin{itemize}
\item \textbf{Auswahl proportional zu Fittnes}, hierbei spielt die im vohrigen Schritt berechnete Fintess eine große Rolle. Die Eltern werden nach dieser Fittness proportional als Elternteil ausgewählt und zum Elternpool hinzugefügt.
\item \textbf{Best 50 prozent},heißt aus der oberen hälfte der alten Generation werden alle Induviduen dem Elternpool hinzugefügt.Aus welchen dann zufällig die einzelnen Elternteile ausgewählt werden. Es mssen natürlich nicht immer 50 prozent sein, es kann sich auch um einen anderen Prozentsatz handeln.
\item \textbf{Tunier Auswahl}, in diesem Verfahren werden k induviduen der Population ausgewählt. Diese k Induviuen Treten dann wie in einem Tunier gegeneinander an. Das Individum mit dem Besten Fittneswert ein Elternteil ausgewählt. Hierbei wird auf den Elternpool verzichtet und direkt ein Kind aus zwei gewinnern erstellt.
\item \textbf{Comma selction}, Genetic algorithm essentails s.17
\item \textbf{Pluss selection}, Genetic algorithm essentails s.17
\end{itemize}


Nun wurden die Eltern ausgewählt um nun aus den Elterngeneration eine neue Kindergeneration zu erstellen, dies wird im nächsten Schritt erklärt.

\paragraph{Paarung aka Breed / Variation}
Aus dem Elternpool werden nun Nachkommen(Kinder) geschaffen. Alleine durch die Paarung(Chrossover) von qualitativ hochwertigen individuen wird erwartet, das die Nachkommen eine bessere Qualität besitzen als die ihrer Eltern.Aber wenn man nur die eigenschaften der Eltern übernimmt gibt es keine Garantie, dass die Kinder eine höhere Qualität erreichen.Es kann sogar dazu kommen, das negative eigenschaften mit übernommen werden. Da dies natürlich nicht gewollt ist gibt es eine einfach Verbesserungs möglichkei. Die Muation, hier wird jedes Gen noch einmal mit einer zufälligen Muation versehen welches ähnliche aber andere Lösungen hervorbringt. Nun gehen wir noch einmal genauer auf Operation Chrossover und Muation ein.


\iffalse
Somit müsste sich die Finttnes der nächsten generation verbessern. Um dies zu ereichen werden die  Gene noch modifiziert, durch corssover oder mutation. Somit wird der suchraum noch einmal vergrößert aber nur in der nähe der für gut empfunden Individuem bzw. dieser Gene.
Um aus den Einzelnen elternpaaren neue Individuen zu generieren wird das Verfahren/algorithmus Crossover verwendet. Bei Crossover kann es nun auch verschiedene möglichkeiten geben.
\fi


\begin{itemize}
\item \textbf{Crossover}, hierfür werden die Chromosome der Kinder Individuen bestimmt. Dazu werden entweder 50 Prozent des ersten Elternteils und 50 Prozent des zweiten elternteils verwendet wie im Oberenteil von Abbildung \ref{fig:chromoson crossover} zu sehen ist. Es gibt auch den Ansatz das ganz zufällig die Gene ausgewählt werden und dem Kind vererbt werden wie im Unterenteil der Abbildung \ref{fig:chromoson crossover} zu sehen ist. 

\begin{figure}[H]
  \centering  
  \includegraphics[scale=0.5]{img/crossover.png}
  \caption{crossover anhand eines einfachen binären Chroms. Das erste zeigt eine 50/50 crossover. Das zweite zeigt eine Zufällige auswahl ders Gens.\cite{Rashid2017} }
  \label{fig:chromoson crossover}
\end{figure}

\item \textbf{Mutation},hierbei wird jedes gens des Indivium zufällig mit einer zufälligen mutation versehen. Durch diese Mutation wird eine neue Inforamtion/Lösung in die nachfolgende Generation übergeben. 

\end{itemize}

\begin{figure}[H]
  \centering  
  \includegraphics[scale=0.5]{img/mutation.png}
  \caption{Muation eines Genes um höhere vielfältigkeit zubekommen.\cite{Rashid2017} }
  \label{fig:chromoson mutation}
\end{figure}


\paragraph{Austausch/ Scheife / Exchange}
Die neue Generation aus Kindern tauscht nun die alte Generation aus. Nun folgen die gleichen Schritte: Grade, Slection, crossover und mutation. 
Diese Schleife wird so lange durchegführt bis die Populationsfittnes das zuvor festgelegte Maximum erreicht. Wenn dies geschied gibt es viele Lösungen welche alle sehr ähnlich sein sollten. Aus dieser kann dann das beste Individum ausgesucht werden und als beste Lösung eingesetz werden. 


\subsection{Verschiedene Arten von GA}
\subsubsection{Standart-GA}
\subsubsection{Steady-State-GA}
\newpage

\subsection{Künstliche Neuronale Netze}

\begin{figure}[htb]
  \centering  
  \includegraphics[scale=0.5]{img/S36_Buildyourown.png}
  \caption{Künstliches Neuronales Netz mit drei Schichten je drei Neuronen \cite{Rashid2017} }
  \label{fig:neural_network}
\end{figure}


Hier zu sehen ist ein Künstliches Neuronales Netz mit drei Schichten \ref{fig:neural_network}. Dies wurde dem natürlichen Vorbild der neuronalen Netze im Gehirn nach empfunden. Die Kreise nennt man Neuronen oder auch Perseptron, mehrere Neuronen zusammen ergeben eine Schicht oder auch Layer genannt. Die Verbindungen repräsentieren die Gewichte, über diese kann einem Netz verschiedene Zusammenhänge von Input und Output antrainiert bzw. angelernt werden. Zum Training werden viele Daten benötigt, aus welchen das Netz \glqq Lernt\grqq{}. Dafür ist es wichtig, viele aufbereitete Daten zu besitzen, denn diese Netze brauchen viele Trainingsiterationen, bis das gewünschte Ergebnis zustande kommt. Ein Neuron besteht aus Eingängen, Gewichten und einer Aktivierungfunktion sowie einem Ausgang. Die Vernetzung mehrerer Neuronenschichten lässt ein Neuronales Netz entstehen.

\newpage



\subsection{Aufbau eines Neurons}
\begin{figure}[htb]
  \centering  
  \includegraphics[scale=0.5]{img/S41_Buildyourown.png}
  \caption{Aufbau eines Neurons \cite{Rashid2017}}
  \label{fig:neuron}
  

\end{figure}
\subsubsection{Eingang aka Input}
Bei dem Input handelt es sich um einfache xxxFloatwert dieser wird mit den einzelnen Gewichten verrechnet. Ein Neuron hat meist mehrere Eingangsgrößen, welche alle zusammen mit den Gewichten aufsummiert werden. Diese Werte werden zufällig initialisiert und per Training verbessert, somit handelt es sich um einen angelernten Werte, welche durch die Backproagation(Fehlerrückführung) verbessert werden.

\subsubsection{Offset aka bias}
Auf dieses Aufsummiertes Ergebniss wird anschließend ein Bias gerechnet, dieser führt zu einem besseren Verhalten beim Trainieren. Bei diesen Werten handelt es sich um angelernte Werte, die per Backpropagation verbessert werden und die Flexibitlität der Netze erhöht.


\subsubsection{Aktivierungs Funktion}
Die Aktivierungsfunktion kann man sich als Schwellwert vorstellen, ab wann das Neuron den Input weiter gibt. Es gibt verschiedene Funktionen, um diesen Schwellwert zu definieren. Je nach Aufgabe des Neuronalen Netze werden andere Aktivierungsfunktionen verwendet. Bei Klassifizierungen werden heute meist ReLu-Layer oder ein Weakly-ReLu Layer benutzt, diese verhindern das Vanishing- bzw. Exploding- gradientproblem beim Trainieren.

\subsubsection{Ausgang aka Output}
Wenn der Schwellwert überschritten wird, wird am Output durchgeschaltet. Dieser Output kann entweder mit einer nen Schicht Neronen verbundne sein oder direkt als Ausgang gesehen werden. Über welchen man anhand von xxxVariabelenwerten/Kommawerten die 
Von Input nach Output nennt sich ein Single-Forward-Pass. Wie hier beschrieben wird, kann ein Netz verschieden viele Layer besitzen mit verschiedenen Anzahlen von Neuronen.

\subsection{Verlustfunktion aka lossfunktion}
Die Verlustfunktion stellt ein ausgesuchtes Maß der Diskrepanz zwischen den beobachteten und den vorhergesagten Daten dar. Sie bestimmt die Leistungsfähigkeit des neuronalen Netzes während des Trainings und der Ausführung. Ziel ist es, im laufenden Prozess der Modellanpassung, die Verlustfunktion zu minimieren.

\subsection{Optimierer alt Gradientenabstieg}
Um die Fehlerfunktion zu minimieren wird als Werkzeug der Gradienten Abstieg benutzt. Diese ist nur möglich da ein Künstliches Neuronales Netz aus verketteten differenzierbaren Gewichte der Neuronen(Tensoroperationen) aufgebaut ist, die es erlauben duch anwendung der Kettenregel die Gradientenfunktion zu finden, die den aktuellen Parametern des Datenstapels werte des Gradienten zuordnet. Es gibt auch hier verschiedene Ansätze von Optimierern, welche die genauen Regeln wie der Gradient der Verlustfunktion zu Aktualisierund der Parameter verwendet wird hier könnte Beispielweise den RMSProp-Optimierer, der die trägheit des Gradientenabstiegsverfahren berücksichtet. Seite 83 - Deep Learning chollet


\iffalse
 Im Grunde werden dabei die Gewichte so angepasst, dass ein besseres Ergebnis entsteht und dadurch die Fehlerfunktion verringert wird. Wie das Wort Backpropagation schon sagt, wird von hinten nach vorne verbessert. Es gibt verschiedene Variationen von Gradientenabstiegen, welche verschiedene Vor- und Nachteile haben. Bei dem Trainieren des Netzes wurde der Momentum-Optimizer, welcher aus einem Gradientenabstieg mit Momentum aufgebaut ist.
\fi

\subsection{Hyperparameter}
Als Hyperparameter werden, in Bezug auf KNN's, meist die Anfangsbedingungen bezeichnet. Somit handelt es sich um die Learnrate (eng. Learningrate), der Abdeckunggrad(eng. Dropout), die verlustfunktion oder auch der Optimizer. In selten fällen kann man die Modelachitektur auch als Hyperparameter bezeichen. Für diese Hyperparameter gelten keine universellen Werte, sondern müssen je nach Daten und Funktion(oder KNN), speziell angepasst und verändert werden. Deshalb gibt es nur einige Regeln und grobe abschätzungen in welchem grenzen sich diese Hyperparameter befinden. 

\subsection{Zusammenfassung}

\section{Stand der Forschung und Technik}
Durch die gestiegene Rechenleistung der heutigen Computer ist auch die anwendungshäufigkeit von Genetischen Algorithmen deutlich gestiegen. Es werden nicht nur neue Ansätze erforscht, sondern auch kleiner Anwendungen entwickelt, die nicht nur im IT-Bereich ihre anwendungen finden.

\subsection{Deep Mind}
Den größten Fortschitt der letzen Jahre verzeichnete Googles Deepmind. Deepmind setzt dabei auch auf einen Populations bassierenden Algorithmus, welcher auf den Genetischen Algorithmen aufbaut. Goolge nutz ihre PopulationBasedTraining vorallem im Bereich der Künstlichen Inteligzenz. So konnten sie bei verschiedene Künstlichen Neuronalen Netzen wie AlphaGo(Go)\cite{alphago} und AlphaStar(Starcraft2)\cite{alphastar} deutliche verbesserungen durch das PBT erreichen. Das PBT wird nicht nur für Brett- bzw. Computer-spiele genutzt. Sondern auch für Anwendungen im Bereich des autonomen Fahrens. Dort wurde in zusammenarbeit mit der Google-Tochter Waymo eine deutliche Verbesserung der xxxgesamt Leistungxxx ausgearbeitet. 

\subsection{PopulationBasedTraining}
PBT ist eine weiter entwicklung der klassischen Grid- und Randomsearch und wird von den Genetischen Grundlagen stark beinflusst. PBT ist wie Ga ein Populations bassierender such algorithmus es werden hier auch weiter methoden wie survival of the Fitest und Muation übernommen. Desweitern wurde ein online anpassung der Hyperparameter wärend des Trainings implementiert. Durch dieses Feature, wird viel Rechenleistung eingespaart bzw. das berechnen der Hyperparameter stark beschleunigt. \cite{pbt}

\iffalse
Dies ist zusehen daran das PBT auch eine Population verwendet, auch werden hier die übernahmme des Fitesten und Muationen verwendet. Zudem ist das Training Asynchron und Parallel möglich.
Desweitern wurde ein online anpassung der Hyperparameter wärend des Trainings implementiert. Durch dieses Feature, wird viel Rechenleistung eingespaart bzw. das berechnen der Hyperparameter stark beschleunigt. \cite{pbt}


Sie verwenden Algorithem die dem GA Sehr ähnlich(weiter entwickelt) ist und zwar das Population bassierende Training (eng. Poulation Based Training short PBT). Sie benutzen PBT auch zum anpassen von Hyperparametern speziel für ihre Reinforcment Learning Models. Deep Minds Variante des GA ist sehr viel komplexer. 
Sie haben ein Online-Learn verfahren in welchem sie die Hyperparameter während des Trainings anpassen können, dies ist durch einen einen Server auf dem die daten gespeichert sind möglich. Dieser gibt ihnen auch die möglichkeit Asynchron und Parallel zu arbeiten. Im Durschnitt kontne sie ihre Ergebnisse noch einmal um bis zu 5 Prozent verbessern. 
\fi

\subsection{Software Testing with Ga}
Die Softwarebewertung spielt eine entscheidende Rolle im Lebenszyklus eines Software-Produktionssystems. Die Erzeugung geeigneter Daten zum Testen des Verhaltens der Software ist Gegenstand vieler Forschungen im Software-Engineering. In diesem Beitrag wird die Qualitätskontrolle mit Kriterien zur Abdeckung von Anwendungspfaden betrachtet und ein neues Verfahren auf der Grundlage eines genetischen Algorithmus zur Erzeugung optimaler Testdaten vorgeschlagen. 
\cite{Keshavarz}


\subsection{Travelling Salesman Problem}
Einer der bekanntesten anwendungen für GA ist das Travelling Salesman Problem, in welchem die kürzeste Route beim Austragen von Briefen für einen Postboten berechnet werden soll. Es ist ein Optimales Problem zum Lösen für GA, durch die zurückgelegte steckte kann leichte eine Fitnessfunktion aufgestellt werden. Desweitern wird der Suchbereich mit jedem auszutragenden Brief um einen Expotonenten größer. Und ist somit für gänige Algorothmen nur mit sehr viel Rechnleistung zu bewältigen.

 Doch je mehr Briefe der Postbote austragen soll umso mehr Variablen gibt es, sprich es wird wesentlich schwerer für ein fest geschrieben (eng. Hardcoded) Algorithmus den kürzesten weg zu finden. Für den Ga ist dies kein Problem da mit der richtigen Fitnessfunktion eine einfacher Rückgabewert der Funktion zu bekommen ist. Dementsprechend kann die Route einfach optimiert werden.

\subsection{Temperatur Schätzungen}
--Fällt weg-- wegen GP 
Es gibt auch Forschungen zur berechnung von Temperaturverläufen der Erde \cite{Stanislawska1}. Vom gleichen Author gibt es auch eine abschätzungen von Wärmeflusses zwischen Athmosphäre und Meereies in Polarregeionen \cite{Stanislawska,2}. All diese Vorraussagen wurden mithilfe von Genetischen Programming ausgerechnet. 

In dem die Aufzeichnungen der Temperaturverläufe als Trainingsdaten verwendet werden konnten Lösungen gefunden werden, welche den rellen daten sehr nahe kommen.

\subsection{Generativ Design}
Heute gibt es in viele Computer-aided design(CAD) Programme in welchen es implementierungen von Generativen Design Werkzeugen gibt. In dennen über Iterationen neue mögliche Designs berechnet werden deis passiert auch auf Basis der Genetischen Evolution. Sie bauen nicht auf den Genetischen Algorithmen auf sind aber nahe verwante und sollten nicht unterschätzt werden. 
Mit ihnen ist es möglich Bionische Stukturen für addetive Vertigung zu designen. Und sie speziell auf die Anwendung anpassen. So kann aus einem einfachen Frästeil ein wesentlich Leichtes und Material spaarenderes Model entwickelt werden.

\noindent%
\begin{figure}[H]
  \centering  
  \includegraphics[scale=0.3]{img/Additive.png}
  \caption{Additives Design über mehrer Iterationen}
  \label{fig:Ablauf_kurz}
\end{figure}

\paragraph{NASA - Antenne}
DIe Nasa hat 2006 eine Weltraumantenne mit hilfe evolutionären Algorithmen entwickelt. Die Entwicklung der Antenne von hand ist sehr zeitaufwändig und arbeitsintehnsiv, zusätzlich braucht man großes wissen in der entsprächenden Domain. Deshalb nutzen die Forscher einen Populations bezogenen Suchansatz, um Umgebungsstrukturen und  elektromagnetischer mit einzubeziehen. Die entwickelte Antenne wurde produziert und auch auf Space Technology 5 mission genutzt.
\cite{AutomatedAntenna}
\noindent%
\begin{figure}[H]
  \centering  
  \includegraphics[scale=0.5]{img/nasa-antenne.png}
  \caption{Foto der Prototypen für unterschiedliche Anforderungen. \cite{AutomatedAntenna}}
  \label{fig:Ablauf_kurz}
\end{figure}


\subsection{GLEAM}
General Learning Evolutionary Algorthm and Method ist eine vom Kit entwickelte Methode um Aktionsketen zu berechnen. Dazu gehörtz zum beispiel das Aufeinandern abstimmen der Maschinen in einem Maschinenpark um so genannte totzeiten der Maschinen zu verringern also die gesamt auslastung zu erhöhen.

Mit GLEAM wurde auch versucht die Stuerung von 6-Achsigen Robotorarmen zu verbessern. Es konnte gezeigt werden das die Steuerung mit Gleam funktioniert, dies wurde aber leider nie der neue Industie Standart. Es wird immer noch mit der klasschischen xxxSteuerungxxx  gearbeitet. 


\subsection{Reainforcment learning with GA}
Reinforcment learning ist möglichweise einer der größten Anwendungsgebiete der GA. Hierbei werden Neuronale Netze nicht mit hilfe von Gradienstieg training, wie im Gundlagen Kapitel besprochen. Sondern mit hilfe von Genetischen Algorithmen. Dabei wird das Neuronale Netz nicht 




\subsection{Zusammenfassung}

\section{Konzept}
\subsection{Einleitung}

\subsection{Anforderungsanalyse}

\subsection{Zusammenfassung}

In diesem Kapitel werden die Implementierungen dieser Arbeit erläutert. Zuerst wird der Systemaufbau erklärt und alle verwendeten Module ausführlich beschrieben. Anschließend wird auf die verwendete Hardware eingegangen. Im Folgenden werden die implementierten Methoden des GA und der Zufallssuche genauer betrachtet, indem die exakten Methoden und künstlichen neuronalen Netze erklärt werden. Zum Schluss wird auf die implementierten Evaluationen und Auswertungen eingegangen. 

\section{Systemaufbau}
Das Gesamtsystem wird mit Python als Programmiersprache umgesetzt. Für die Implementierung werden zusätzliche Python Pakete verwendet. 
Für das Erstellen und Trainieren der künstlichen neuronalen Netze wird auf Tensorflow zurückgegriffen. Tensorflow ist eine End-to-End-Open-Source Plattform für maschinelles Lernen (ML). Zudem bietet Tensorflow ein umfassendes und flexibles Ökosystem aus Werkzeugen und Bibliotheken für ML-Anwendungen. Darüber hinaus hat Tensorflow eine starke Community, welche vorallem im ML-Bereich viele Open-Source-Projekte entwickelt und veröffentlicht.
Für die Zufallssuche wird SkLearn verwendet. Bei SkLearn handelt es sich um ein simples und effizientes Werkzeug zur prädiktiven Datenanalyse. Dies beinhaltet Funktionen wie:  Klassifikation, Regression, Clustering, Support Vektor Maschine. Mit Hilfe von SkLearn werden die bewertungsmatrixen zum Auswerten der Algorithmen umgesetzt.
Zum Darstellen von Diagrammen wird Seaborn und Matplotlib verwendet.
Der Genetische Algorithmus wurde eigenständig implementiert. Für diesen wurden nur die Pakete Numpy und Pandas verwendet. Numpy ist ein Paket für Multi-Dimensionen Arrays und bietet eine große Anzahl an mathematischen Funktionen. Pandas ist wie Numpy für Arrays und Datenframes zuständig. Im Detail ist Pandas für Datenstrukturen und Manipulation solcher Datenarrays entwickelt worden.
Da die Rechenzeit ein entscheidender Faktor der Optimierung darstellt, wird das Paket Multiprocessing verwendet. Dies führt vor allem bei Multi-Core-Servern zu schnelleren Berechnungen. Das Gesamtsystem wurde für CPU entwickelt. Darüber hinaus ist der Genetischen Algorithmus als GPU (engl. graphics processing unit) optimierte Implementierung vorhanden. Dies beschleunigt die Berechnungen von großen KNNs deutlich. 
All diese Einstellungen können per Argumente an das Python Skript übergeben werden. Somit ist es einfach möglich andere Berechnungen durchzuführen, auch solche, die in dieser Arbeit nicht besprochen werden. Eine ausführliche Auflistung aller verwendeten Pakete ist in Abbildung \ref{fig:Python_pakete} zu sehen. 

\begin{figure}[H]
  \centering  
  \includegraphics[scale=1]{img/Python_pakete.pdf}
  \caption{Datenfluss und verwendete Python Pakete, die zur Umsetzung der Arbeit verwendet wurden}
  \label{fig:Python_pakete}
\end{figure}

\subsection{Hardware}
Um die Optimierungen durchzuführen stehen drei Hardwaremöglichkeiten zur Auswahl. Dazu gehört ein Server mit i5-5550 Prozessoren mit 40 Kernen ohne GPU, ein Server mit einer Grafikkarte (Gtx1080Ti) um die Berechnungen durchzuführen und eine Server mit einem i7-7700 Prozessor mit 8 Kernen ohne GPU. Diese sind tabellarisch in Tabelle \ref{tab:Server} aufgeführt. Im späteren Kapitel \ref{sec:Evaluierung} wird auf die Benchmarks der Hardware eingegangen.

\begin{table}[htb]
\centering
\caption{Zur Verfügung stehende Hardware} \label{tab:Server}
\begin{tabular}{lccc}\toprule
\textbf{Hardwarekomponenten}	&\textbf{Server 1} &\textbf{Server 2} &\textbf{Server 3}	\\\midrule
Prozessor		& E5-2630   & i7-7700	& i7-7700 \\
Logische Kerne 	& 40        & 8     	& 8 	\\
Arbeitsspeicher	& 256 GB    & 8 GB	    & 32 GB	\\
Grafikkarte		& -	        & -	        & GTX 1080 Ti (11 GB)	\\\bottomrule
\end{tabular}
\end{table}


\subsection{Datensatz}
Um die künstlichen neuronalen Netze zu trainieren wurden Bilddaten als Daten-Grundlage ausgewählt. Bei den verwendeten Datensätzen handelt es sich um gängige Datensätze, die oft zum Vergleichen von Klassifikationsergebnissen von KNN genutzt werden. Folgende Datensätze wurden ausgewählt: Mnist Faschion, Mnist Digits und Cifar10. Diese drei Datensätze werden benutzt, da sie sich sehr ähnlich sind. Alle drei besitzen eine gleichmäßige Verteilung der Daten in Bezug auf die Klassen. Die Datensätze besitzen jeweils 10 Klassen. Die Besonderheit der Datensätze liegt auf ihrer guten Vergleichbarkeit. Die einzigen Unterschiede sind die Klassen und Bildformate. Alle verwendeten Datensätze werden wie folgt aufgeteilt: 65\% Trainingsdaten, 20\% Validierungsdaten, 15\% Testdaten. Somit wird das Overfitten der Netze verhindert, da die Daten klar getrennt werden. Nachfolgend werden die Datensätze genauer beschrieben.

\paragraph{Mnist Fashion Datensatz}
Zur Evaluierung der Methoden des Genetischen Algorithmus wurde der "`Mnist Fashion"' Datensatz benutzt. Dieser Datensatz enthält 10 Klassen für verschieden Kleidungsstücke. Er beinhaltet die Klassen: T-shirt/Top, Trouser, Pullover, Dress, Coat, Sandal, Shirt, Sneaker, Bag, Ankle Boot. Der Mnist Fashion Datensatz besteht aus 70.000 Schwarz-weiß Bildern mit je 28x28 Pixeln. Ein Beispiel aus dem Datensatz ist in Abbildung \ref{fig:dataset_example} auf der linken Seite zusehen. 

\paragraph{Mnist Digits Datensatz}
Zum Evaluieren der Optimierung des Fully Conneceted Netzes wird der "`Mnist Digits"' Datensatz benutzt. Dieser besteht aus handgeschrieben Ziffern zwischen 0 und 9. Der Mnist Digits Datensatz besteht aus 70.000 Schwarz-weiß Bildern mit je 28x28  Pixeln. Ein Beispiel des Mnist Digits Datensatze ist in der Mitte der Abbildung \ref{fig:dataset_example} zusehen. Um die Evaluation der Optimierung nicht nur mit einem Datensatz durchzuführen, wird ein dritter Datensatz verwendet.

\paragraph{Cifar 10 Datensatz}
Hierbei handelt es sich um den CIFAR 10 Datensatz. Er besitzt die Klassen: Flugzeug, Auto, Vogel, Katze, Reh, Hund, Pferd, Schiff und LKW. Der Datensatz ist aus 60.000 farbigen Bildern (RGB) mit je 32x32 Pixel aufgebaut. Ein Beispiel ist auf der rechten Seite der Abbildung \ref{fig:dataset_example} zusehen. 

\begin{figure}[H]
  \centering  
  \includegraphics[scale=0.7]{img/dataset_example.pdf}
  \caption{Beispiele aus den verwendeten Datensätze}
  \label{fig:dataset_example}
\end{figure}

\section{Genetischer Algorithmus} \label{ssec:implementierung}
Die Implementierung des Genetischen Algorithmus entspricht den 5 Schritten des Genetischen Algorithmus, das Konzept wurde in Kapitel \ref{ssec:GA} beschrieben. In den folgenden Unterkapitel werden die implementierten Methoden im Detail besprochen. Zusätzlich wird auf speziellen Umsetzungen und Eigenschaften eingegangen. Da es zu den meisten Methoden in der Literatur wenig Informationen zu ihren Auswirkungen auf das Endergebnis gibt, wird ein Experiment zur Bestimmung der zielführenden Methoden durchgeführt. Für dieses Experiment wurde eine Optimierung der Hyperparameter durchgeführt. Als Netz wird ein Fully-Connected Netz mit einem Layer und 128 Neuronen verwendet, die Größe des Netzes wird im Abschnitt \ref{tab:fully_small} ausführlich beschrieben. Es wurde diese Größe gewählt, da die spätere Evaluation mit der gleichen Größe an Netz durchgeführt wird. Als Datengrundlage wurde der Mnist Fashion Datensatz verwendet, linke Seite der Abbildung \ref{fig:dataset_example}. Die Methoden mit den die besten Ergebnisse gefunden wurden, werden zur Evaluierung des Genetischen Algorithmus eingesetzt. Das Wissen zu den Grundlagen der Methoden des Genetischen Algorithmus wird für dieses Kapitel als voraussetzt angenommen und kann im Grundlagenkapitel \ref{Genetische_Algorithmen} nachgelesen werden.

\newpage

\subsection{Initialisierung der Anfangspopulation} \label{implementierung_init}
Die Größe der Anfangspopulation kann beliebig bestimmt werden. Im späteren Evaluationsabschnitt \ref{sec:Evaluierung} wird die genauen Größe definiert, da diese im direkten Zusammenhang mit der Laufzeit steht. Der Suchraum in dem die Hyperparameter zufällig initialisiert werden, ist in Tabelle \ref{tab:Rahmen} zusehen. Die Grenzen des Suchraums wurde mithilfe der Literatur und des beschriebenen Experiments bestimmt. Die Initialisierung im Suchraum ist zufällig mit einer gleichmäßigen Verteilung umgesetzt. Diese ist als Formel in Eq. \ref{eq:10} abgebildet.

\begin{equation}
	p(x) = \frac{1}{b - a}  \label{eq:10}
\end{equation}
a und b bilden die Intervallgrenzen.

\begin{table}[htb]
\centering
\caption{Suchraum des Genetischen Algorithmus}\label{tab:Rahmen}
\begin{tabular}{lccc}\toprule
\textbf{Hyperparameter}	&\textbf{Minimum}   &\textbf{Maximum}	\\\midrule
Prozessor		        & 0.000005          & 0.1 \\
Logische Kerne 	        & 0.05          	& 0.5 \\
Arbeitsspeicher     	& 10                & 80	\\
Grafikkarte		        & 0	                & 7	\\\bottomrule
\end{tabular}
\end{table}
Hierbei steht jede ganze Zahl beim Optimierer für einen eigenen Optimierer. Hierbei entspricht: 0 = adam, 1 = SGD, 2 = RMSprop, 3 = Adagrad, 4 = adadelta, 5 = adammax, 6 = nadam, 7 = ftrl


\subsection{Fitnessfunktion}\label{implementierung_Fitnessfunktion}
Die Fitnessfunktion beinhaltet den rechenaufwendigsten Teil der Arbeit. In dieser werden die KNNs berechnet. Dazu werden die Gene (Hyperparameter) als Input verwendet. Anschließend werden die Daten geladen, das Netz trainiert und anschließend evaluiert. Der Rückgabewert ist die Klassifizierungsgenauigkeit, die dem Fitnesswert entspricht. Die Fitnessfunktion ist in Abbildung \ref{fig:Fitnessfunktion} als Whitebox dargestellt. Die Fintessfunktion wird für unterschiedliche KNN erstellt. Diese Netze sind im Kapitel \ref{sec:Evaluierung} beschreiben.

\noindent%
\begin{figure}[H]
  \centering  
  \includegraphics[scale=1]{img/Fitnessfunction.pdf}
  \caption{Beispielhafte Whitebox der Fitnessfunktion mit Genen als Input und Klassifikationsgenauigkeit als Output}
  \label{fig:Fitnessfunktion}
\end{figure}  


\subsection{Selektion der Eltern} \label{implementierung_Selektion_Eltern}
Aus dem Experiment hat sich gezeigt, dass bei der Selektion der Eltern die Fitness proportional Selektion(FPS) die besseren Ergebnisse im Vergleich zu der Ranking Selektion liefert. Aus diesem Grund wird die FPS bei der Evaluation des Genetischen Algorithmus verwendet.

\subsubsection{Vermehrung} \label{implementierung_vermehrung}
Für das \textbf{Crossover} wurde Two-Point Crossover und Uniform-Crossover implementiert. Für die \textbf{Mutation} wurde die Gauss-Mutation mit der Gaussdichteverteilung, mit Sigma gleich 0.1 (Eq. \ref{eq:11}) und eine Mutation mit gleichmäßiger Verteilung (Eq. \ref{eq:10}) implementiert. Für die spätere Evaluation der Optimierung mit dem GA wurde Two-Point Crossover und die Gauss-Mutation ausgewählt, da sie in den Experimenten die zielführendsten Ergebnisse lieferten.

\begin{equation}
	p(x) = \frac{1}{\sqrt{ 2 \pi \sigma^2 }} e^{ - \frac{ (x - \mu)^2 } {2 \sigma^2} } \label{eq:11}
\end{equation} 

\newpage

\subsection{Neue Generation}
Die Population der neuen Generation wird mit den genannten Methoden ausgewählt. Diese sind in Abbildung \ref{fig:Ga_Methoden} noch einmal dargestellt. Die Experimente zeigten, dass die Optimierungsergebnisse sich steigern lassen, wenn die besten zwei Individuen der alten Generation, ohne Mutation und Crossover, in die nächste Generation übergeben werden. Dies kann für beliebig viele Generationen durchgeführt werden oder bis zum Erreichen einer Abbruchbedingung.

\noindent%
\begin{figure}[H]
  \centering  
  \includegraphics[scale=0.9]{img/Ga_Methoden.pdf}
  \caption{Implementierte und verwendete Methoden des Genetischen Algorithmus}
  \label{fig:Ga_Methoden}
\end{figure}

\section{Zufallssuche}
Die Zufallssuche wird mit Hilfe der Python Bibliothek SkLearn umgesetzt. Die implementierte Zufallssuche ist, wie im Grundlagenkapitel \ref{Zufalls_Suche} beschrieben, aufgebaut. Für den Suchraum der Zufallssuche gelten die gleichen Rahmenbedingungen wie für den Genetischen Algorithmus. Im Vergleich zum GA wird der Suchraum bei der Zufallssuche in ein Raster aufgeteilt. Dieses Raster ist in Tabelle \ref{tab:Raster} zusehen.


\begin{table}[htb]
\caption{Suchraum der Zufallssuche} \label{tab:Raster}
\begin{tabular}{lccclllll}\toprule
\textbf{Hyperparameter} &\textbf{Min}   &   &   &   &   &  &   &\textbf{Max}	\\\midrule
Lernrate       & 0.00005 & 0.0001 & 0.0005 & 0.001 & 0.005 & 0.01 & 0.05 & 0.1  \\
Dropout        & 0.05    & 0.1    & 0.2    & 0.3   & 0.4   & 0.5  & 0.6  & 0.7  \\
Epochen        & 10      & 20     & 30     & 40    & 50    & 60   & 70   & 80   \\
Batchsize      & 8       & 16     & 32     & 40    & 48    & 56   & 64   & 72   \\
Optimizer      & 0       & 1      & 2      & 3     & 4     & 5    & 6    & 7    \\\bottomrule
\end{tabular}
\end{table}

Jede ganze Zahl des Optimizer steht für einen eigenen Optimierer. Hierbei entspricht 0 = adam, 1 = SGD, 2 = RMSprop, 3 = Adagrad, 4 = adadelta, 5 = adammax, 6 = nadam, 7 = ftrl.

\section{Evaluierung} 
\label{sec:Evaluierung}
Zur Evaluierung wird die Berechnungszeit der beiden Algorithmen gleichgesetzt. Da das Trainieren der KNN 95\% der Berechnungszeit der Algorithmen ausmacht, wird die Berechnungszeit über die Anzahl der Trainingsvorgänge begrenzt. Durch Versuche ergab sich die Evaluierung von 50 und 250 Iterationen als zielführend. Eine weitere Verdopplung der Iterationen von 250 auf 500, zeigte bei den Versuchen keine Verbesserung der Ergebnisse. Trotz des doppelten Rechenaufwandes. Aus diesem Grund ist die Berechnung von 500 Iterationen nicht umgesetzt worden. Für die Zufallssuche ergibt sich somit 50 und 250 Iterationen. Für den Genetischen Algorithmus ergibt sich für 50 Iterationen eine Populationsgröße von 25 Individuen mit 2 Generationen. Für 250 Iterationen wird im GA eine Populationsgröße von 50 Individuen und 5 Generationen gewählt. Um aussagekräftige Ergebnisse zu erhalten wurden zwei KNNs implementiert:


\subsection{Hyperparameter Optimierung - Fully Connected Network}
Es wird ein Fully Connected Network zur Klassifizierung von handgeschriebenen Ziffern implementiert. Darüber hinaus werden zwei verschiedene Größen des künstlichen neuronalen Netzes implementiert, um mögliche Unterschiede der Modellgröße auf die Optimierung zu erkennen. Für das Fully-Connected-Network wurden diese Größen von KNNs implementiert:
\begin{itemize}
\item \textbf{kleines KNN} mit einem Layer und 128 Neuronen. Die exakte Modell-Architektur ist in der nachfolgenden Auflistung zusehen. Die 28x28 Pixel der Eingangsdaten werden zu 784 Pixel mit nur einer Dimension flachgedrückt (engl. flatted). Anschließend kommt die Versteckteschicht (engl. hidden Layer) mit 128 Neuronen und einer  Dropoutschicht, die für jedes Neuron durchgeführt wird. Abschließend kommt die Ausgangsschicht mit 10 Klassen, wodurch sich 10 Neuronen in der Ausgangsschicht ergeben. Dies entspricht einer Anzahl von 101.770 trainierbaren Parametern. Somit ist dieses Netz im Vergleich zu State-of-the-Art Netzen relativ klein. Diese Netz ist beispielhaft als Figur \ref{fig:mlp_128} und als Auflistung in der Liste \ref{auflistung_fully_klein} zusehen.

\noindent%
\begin{figure}[H]
  \centering  
  \includegraphics[scale=0.9]{img/mlp_128.pdf}
  \caption{Beispielhafte Darstellung des kleinen KNNs}
  \label{fig:mlp_128}
\end{figure}


Beispielhaft wird die Trainingsdauer des kleinen Netzes für 10 und 80 Epochen evaluiert. Bei 10 Epochen dauerte der Trainingsvorgang 34 Sekunden und für 80 Epochen 252 Sekunden. Diese Werte wurden auf einem Intel i7-7700 3,6 GHz erreicht.
\newpage
{\small
\begin{lstlisting}[language=C,caption=Exakte Modell-Architektur des kleinen Fully-Conneted-Networks,label=auflistung_fully_klein]
_______________________________________________________________
Layer (type)                 Output Shape              Param #
===============================================================
flatten (Flatten)            (None, 784)               0
_______________________________________________________________
dense (Dense)                (None, 128)               100480
_______________________________________________________________
dropout (Dropout)            (None, 128)               0
_______________________________________________________________
dense_1 (Dense)              (None, 10)                1290
===============================================================
Total params: 101,770
_______________________________________________________________
\end{lstlisting}
}

\item \textbf{großes KNN} mit zwei Layer je 128 Neuronen. Die exakte Modell-Architektur ist in der nachfolgenden Auflistung zusehen. Die 28x28 Pixel der Inputdaten werden zu 784 Pixel mit nur einer Dimension flachgedrückt. Anschließend kommt die erste Versteckteschicht mit 128 Neuronen und einer Dropoutschicht, die für jedes Neuron durchgeführt wird. Nun kommt erneut eine Versteckteschicht und Dropout mit 128 Neuronen. Abschließend kommt die Ausgangsschicht mit 10 Klassen, wodurch sich 10 Neuronen in der Ausgangsschicht ergeben. Dies entspricht einer Anzahl von 134.794 trainierbaren Parametern. Somit ist das Netz im Vergleich zu State-of-the-Art Netzen relativ klein. Diese Netz ist beispielhaft als Figur \ref{fig:mlp_2x128} und als Auflistung in der Liste \ref{auflistung_fully_groß} zusehen.

\noindent%
\begin{figure}[H]
  \centering  
  \includegraphics[scale=0.9]{img/mlp_2x128.pdf}
  \caption{Beispielhafte Darstellung des großen KNNs}
  \label{fig:mlp_2x128}
\end{figure}

Beispielhaft wird die Trainingsdauer des großen Netzes für 10 Epochen und 80 Epochen evaluiert. Bei 10 Epochen dauerte der Trainingsvorgang 44 Sekunden und für 80 Epochen 339 Sekunden. Diese Werte wurden auf einem Intel i7-7700 3,6 GHz erreicht. 

{\small
\begin{lstlisting}[language=C,caption=Exakte Modell-Architektur des großen Fully-Conneted-Networks,label=auflistung_fully_groß]
_______________________________________________________________
Layer (type)                 Output Shape              Param #
===============================================================
flatten (Flatten)            (None, 784)               0
_______________________________________________________________
dense (Dense)                (None, 128)               100480
_______________________________________________________________
dropout (Dropout)            (None, 128)               0
_______________________________________________________________
dense_1 (Dense)              (None, 128)               16512
_______________________________________________________________
dropout_1 (Dropout)          (None, 128)               0
_______________________________________________________________
dense_2 (Dense)              (None, 128)               16512
_______________________________________________________________
dropout_2 (Dropout)          (None, 128)               0
_______________________________________________________________
dense_3 (Dense)              (None, 10)                1290
===============================================================
Total params: 134,794
_______________________________________________________________
\end{lstlisting}
}


\end{itemize}

\subsection{Hyperparameter Optimierung - Convolutional Neural Network}
Es wurde ein faltendes neuronales Netz (engl. Convolutional Neural Network - CNN) zur Klassifizierung von kleinen RGB Bildern aus dem Cifar10 Datensatz implementiert. Die exakte Modell-Architekturen ist in der nachfolgenden Auflistung zusehen.
Beim CNN werden die Pixel nicht flachgedrückt, sondern können als Matrix geladen werden. Anschließend kommen verschiedene Kombinationen aus Convolutionalschicht, Activationschicht, Dropoutschicht und Maxpoolingschicht. Diese Schichten besitzen deutlich mehr Parameter. Dieses CNN besitzt 1.250.858 trainierbare Parameter und besitzt damit fast 10 Fach so viele trainierbaren Parametern wie die FCN.

Beispielhaft wird die Trainingsdauer des CNN-Netzes für 10 Epochen und 80 Epochen evaluiert. Bei 10 dauerte der Trainingsvorgang 15 Minuten und für 80 Epochen 125 Minuten. Diese Werte wurden auf einem Intel i7-7700 3,6 GHz erreicht. 

{\small
\begin{lstlisting}[language=C,caption=Exakte Modell-Architektur des Convolutional Neural Network ,label=auflistung_cnn]
_______________________________________________________________
Layer (type)                 Output Shape              Param #
===============================================================
conv2d (Conv2D)              (None, 32, 32, 32)        896
_______________________________________________________________
activation (Activation)      (None, 32, 32, 32)        0
_______________________________________________________________
conv2d_1 (Conv2D)            (None, 30, 30, 32)        9248
_______________________________________________________________
activation_1 (Activation)    (None, 30, 30, 32)        0
_______________________________________________________________
max_pooling2d (MaxPooling2D) (None, 15, 15, 32)        0
_______________________________________________________________
dropout_1 (Dropout)          (None, 15, 15, 32)        0
_______________________________________________________________
conv2d_2 (Conv2D)            (None, 15, 15, 64)        18496
_______________________________________________________________
activation_2 (Activation)    (None, 15, 15, 64)        0
_______________________________________________________________
conv2d_3 (Conv2D)            (None, 13, 13, 64)        36928
_______________________________________________________________
activation_3 (Activation)    (None, 13, 13, 64)        0
_______________________________________________________________
max_pooling2d_1 (MaxPooling2 (None, 6, 6, 64)          0
_______________________________________________________________
dropout_2 (Dropout)          (None, 6, 6, 64)          0
_______________________________________________________________
flatten_1 (Flatten)          (None, 2304)              0
_______________________________________________________________
dense_2 (Dense)              (None, 512)               1180160
_______________________________________________________________
activation_4 (Activation)    (None, 512)               0
_______________________________________________________________
dropout_3 (Dropout)          (None, 512)               0
_______________________________________________________________
dense_3 (Dense)              (None, 10)                5130
_______________________________________________________________
activation_5 (Activation)    (None, 10)                0
===============================================================
Total params: 1,250,858
_______________________________________________________________
\end{lstlisting}
}

\subsection{Hyperparameter Optimierung - kleiner Datensatz}
Bei der Optimierung mit kleinen Datensätzen wird geprüft, ob sich andere Ergebnisse beim Optimieren der Hyperparameter mit kleinerem Datensatz zeigen. Dazu werden die gleichen Datensätze und Netze wie in den vorherigen Evaluierungen verwendet. Nur wird in diesem Fall, die Trainingsdatensätze um 90\% der vorhanden Daten verkleinert. Der Anteil an Testdaten bleibt gleich, so werden die Endergebnisse auf dem gleichen Daten evaluiert.

\subsection{Versuch der Modell-Architektur Optimierung}
Abschließend wird ein Versuch zur Optimierung der Modell-Architektur eines klassifizierungs Netzes implementiert. Für diese Optimierung wird ein Fully-Connected-Network verwendet. Zur Umsetzung der Modell-Architektur Optimierung werden die Neuronen pro Schicht als Gene umgesetzt, zusehen in Abbildung \ref{fig:gene_neuronen}. In diesem Versuch werden die Gene mit drei Schichten umgesetzt. Für die Neuronen pro Schicht, wird ein Minimum von 10 Neuronen und ein Maximum von 256 Neuronen, festgelegt. Mithilfe des Versuches soll eine optimale Modell-Architektur gefunden werden. Es werden die Hyperparameter des ersten Versuchs verwendet.

\begin{figure}[H]
  \centering  
  \includegraphics[scale=1.5]{img/gene_neuronen.pdf}
  \caption{Beispielhafte Zeichnung des Chromosom eines Individuums in welchem die Neuronen als Gene realisiert wurden}
  \label{fig:gene_neuronen}
\end{figure}

\subsection{Auswertung}
Alle Ergebnisse werden automatisch mit den dazugehörigen Konfigurationen und Berechnungen in einer Json-Datei abgespeichert. Aus dieser Json-Datei wird dann automatisch eine Zusammenfassung aller Ergebnisse erstellt und anschließend in einer Excel Datei abgespeichert. Des Weiteren werden Dichte Diagramme aus den Json-Dateien zu den einzelnen Hyperparametern automatisch erstellt und gespeichert. In diesen werden die Dichteverteilungen der einzelnen Hyperparameter in Bezug auf die Fitness dargestellt. Somit können die Hyperparameter intuitiv ausgewählt werden. Der Ablauf der Auswertung ist schematisch in Abbildung \ref{fig:implementierung_auswertung} abgebildet. 

\begin{figure}[H]
  \centering  
  \includegraphics[scale=0.9]{img/Auswertung.pdf}
  \caption{Datenstruktur der Auswertung}
  \label{fig:implementierung_auswertung}
\end{figure}


\section{Zusammenfassung}
In diesem Kapitel wurde auf die implementierten Optimierungsalgorithmen und künstlichen neuronalen Netzen eingegangen. Dazu wurde zuerst der Systemaufbau, Programmiersprache und alle dazu gehörigen Python-Pakete und ihre jeweilige Funktion erläutert. Nachfolgend wurde auf die Hardware, die zur Verfügung stand, eingegangen. Danach wurden die verwendeten Datensätze: Mnist Fashion, Mnist Digits und der Cifar10 Datensatz erklärt. Bei allen drei handelt es sich um Bilddatensätze, welche mit 28x28 Pixel und 32x32x3 Pixel, relativ kleine Bilddaten enthalten. Anschließend wird erläutert, welche 5 Schritte für den Genetischen Algorithmus implementiert wurden, sowie die Methoden, die zur Evaluierung des GA eingesetzt werden. Ebenso wird die implementierte Zufallssuche erläutert welche zum Vergleich des GA dient. Beide Algorithmen benutzen die gleichen Datensätze, als auch die gleichen künstlichen neuronalen Netze. Als KNNs werden ein Fully Conneted Netz mit zwei Variationen optimiert und ein Convolutional Neuronal Network verwendet. Zum Abspeichern der Netze werden Json und Excel Files benutzt. Aus diesen Files werden dann die Auswertungen bzw. das Dichte-Diagramm bestimmt.

In diesem Kapitel wird auf die Ergebnisse der Evaluation eingegangen. Zudem wird auf die Benchmarks der CPU und GPU Versionen eingegangen. Anschließend werden die Ergebnisse der Hyperparameter Optimierung von Fully-Connected-Networks und Convolution-Neuronal-Networks ausgewertet und besprochen. Abschließend wird auf automatischen Dichte-Diagramme eingegangen, dazu wird eine Optimierung beispielhaft ausgewählt und erklärt. 

\section{Benchmarks eines Trainingsvorganges auf der CPU und GPU}
Zur Berechnung der Trainingsvorgänge wurden eine GPU und eine CPU optimierte Variante implementiert. Generell ist die Berechnung von künstlichen neuronalen Netzen mit der GPU wesentlich schneller als mit der CPU. Dies ist aber nicht immer der Fall, weshalb Tests durchgeführt wurden um sie nach ihrer Performance einzustufen. Um eine verbesserte Vergleichbarkeit zu erhalten wird kein Multiprocessing verwendet, da dieses auf der GPU nicht bzw. nur bedingt möglich ist. In der Tabelle \ref{tab:benchmarks} sind die erreichten Benchmarks abgebildet. Die Benchmarks wurden für das kleine Fully-Connected-Network(FCN) \ref{auflistung_fully_klein} und das Convolutional-Neural-Network(CNN) \ref{auflistung_cnn} erstellt. Für das FCN war die Berechnung mit der CPU deutlich schneller als die Berechnungen auf der GPU. Beim CNN zeigte sich das genaue Gegenteil. Die GPU schnitt, mit 270 Sekunden, deutlich besser gegenüber der CPU, mit 470 Sekunden, ab. Dies ist auf die Anzahl der trainierbaren Parameter zurückzuführen. Somit konnte nachgewiesen werden, dass das Training auf der CPU für kleine Fully-Connected-Network effektiver ist, zudem kann auf der der CPU noch Multiprocessing verwendet werden. Dies kann das Training zusätzlich beschleunigen. Im Gegensatz dazu stehen die größeren Netze, wie das Convolutional-Neuronal-Network, dieses trainiert deutlich schneller auf der GPU. Dies ist auf die Größere der Daten und trainierbaren Parametern zurück zuführen. Da die Trainingsdaten des CNNs dreimal so groß \ref{fig:dataset_example} sind. Zudem besitzt das CNN 10 mal so viele trainierbaren Paramter wie das FCN \ref{sec:Evaluierung}. Aus diesem Test hat sich ergeben, dass die kleinen Netze auf der CPU mit Multiprocessing und die großen Netze auf der GPU trainiert werden.


\begin{table}[h]
\centering
\caption{Benchmarks für den Trainingsvorgang der Netze}
\label{tab:benchmarks}
\begin{tabular}{lll} \toprule
Time to Train 10 Epochs & GPU     & CPU      \\\midrule
FCN                     & 160 sec & 27 sec   \\
CNN                     & 260 sec & 470 sec \\\bottomrule
\end{tabular}
\end{table}


\section{Ergebnisse der Evaluation}
Für jede durchgeführte Optimierung wird eine Tabelle erstellt, in welcher die Evaluationsergebnisse festgehalten werden. Diese Ergebnisse beinhalten die folgenden Bewertungsmetriken: Klassifizierungsgenauigkeit, Precision, Recall und F1-Score und werden folgenderweise definiert:
\begin{itemize}
\item \textbf{Klassifizierungsgenauigkeit}
(eng. accuracy) wird wie in Gleichung \ref{eq:12} definiert.
\begin{equation}
Accuracy = \frac{True\ Positive + True \ Negativ}{True\ Positive + True\ Negative + False\ Positive + False\ Negative}
\label{eq:12}
\end{equation}

\item \textbf{Precision} wird wie in Gleichung \ref{eq:13} definiert. 

\begin{equation}
Precision = \frac{True\ Positive}{True\ Positive + False\ Positive} \label{eq:13}
\end{equation} 

\item \textbf{Recall} wird wie in Gleichung \ref{eq:14} definiert. 

\begin{equation}
Recall = \frac{True\ Positive}{True\ Positive + False\ Negative} \label{eq:14}
\end{equation} 

\item \textbf{F1 Score} wird wie in Gleichung \ref{eq:15} definiert. 

\begin{equation}
F1{\text -}Score = 2\frac{Recall\cdot Precision}{Recall+Precision}\label{eq:15}
\end{equation} 
\end{itemize}

Diese Bewertungsmetriken werden für das beste künstliche neuronale Netz jedes Optimierungsalgortihmus berechnet. Anhand diesen Metriken werden die Algorithmen anschließend verglichen. Diese vier Metriken werden später in der gleichen Formation, wie in Tabelle \ref{tab:example_eval}, dargestellt. Die Ergebnisse werden in vier Abschnitte gegliedert: Ergebnis der Hyperparameter Optimierung eines Fully-Connected-Network, Ergebnis der Hyperparamet Optimierung eines Convolutional Neural Network, Ergebnis der Hyperparamter Optimierung mit verkleinertem Datensatz und Ergebnis des Versuchs zur Modell-Architektur Optimierung.


\begin{table} [h]
\centering
\caption{Aufbau der Evaluationsergebnisse}
\label{tab:example_eval}
\begin{tabular}{lll}\toprule
Größe des KNN                & \multicolumn{2}{l}{Größe des trainierten Netzes}       \\\midrule
\multirow{2}{*}{Algorithmus} & accuarcy     & precision score  \\
                             & recall score & f1 score        \\\bottomrule
\end{tabular}
\end{table}


\subsection{Ergebnis der Hyperparameter Optimierung eines Fully-Connected-Network}
Die Ergebnisse der Optimierung der Hyperparameter eines Fully-Connected Netzes für 50 Iterationen ist in Tabelle \ref{tab:fully_50} zusehen und für 250 Iterationen, in Tabelle \ref{tab:fully_250}. In diesen Tabellen werden die Metriken für zwei unterschiedlich große Fully-Connected Netze berechnet. Beide Netze wurde auf den Mnist Digits Datensatz trainiert und getestet. Darüber hinaus werden die Hyperparameter, wie in Kapitel \ref{sec:Evaluierung}, beschreiben für 50 und 250 Iterationen optimiert. Die Ergebnisse der optimierten Netze sind sich sehr ähnlich. Zudem ist der Nachkommaberreich vernachlässigbar, somit bringt der Genetische Algorithmus keine effektive Verbesserung der Ergebnisse gegenüber der Zufallssuche. Auch bei höheren Iterationen ist keine Verbesserung der Ergebnisse mithilfe des GA zuerkennen. Die Größe der KNN macht keine Unterschiede auf das Klassifizierungsergebnis. Darüber hinaus sind keine Unterschiede in den Bewertungsmetriken zu erkennen. 


\begin{table}[h]
\centering
\caption{Ergebnisse der Algorithmen auf dem Mnist Digits Datensatz für 50 Iterationen}
\label{tab:fully_50}
\begin{tabular}{lllll} 
\toprule
Größe des KNN & \multicolumn{2}{l}{klein} & \multicolumn{2}{l}{groß}  \\ 
\midrule
Genetischer Algorithmus            & 98,13\% & 98,20\%         & 97,94\% & 97,86\%         \\
              & 98,19\% & 98,19\%         & 97,84\% & 97,85\%         \\
Zufallssuche            & 97,87\% & 97,82\%         & 97,99\% & 98,23\%         \\
              & 97,80\% & 97,81\%         & 98,22\% & 98,23\%         \\
\bottomrule
\end{tabular}
\end{table}

\begin{table}
\centering
\caption{Ergebnisse der Algorithmen auf dem Mnist Digits Datensatz für 250 Iterationen}
\label{tab:fully_250}
\begin{tabular}{lllll} 
\toprule
Größe des KNN & \multicolumn{2}{l}{klein} & \multicolumn{2}{l}{groß}  \\ 
\midrule
Genetischer Algorithmus             & 98,04\% & 97,98\%         & 98,08\% & 98,00\%         \\
              & 97,98\% & 97,98\%         & 97,94\% & 97,96\%         \\
Zufallssuche            & 98,03\% & 98,24\%         & 98,18\% & 98,14\%         \\
              & 98,23\% & 98,23\%         & 98,11\% & 98,11\%         \\
\bottomrule
\end{tabular}
\end{table}




\subsection{Ergebnis der Hyperparameter Optimierung eines Convolutional Neural Network}
Die Ergebnisse der Optimierung der Hyperparameter eines Covolutional Neural Network sind in Tabelle \ref{tab:cnn} zusehen. Es haben sich ähnliche Ergebnisse für das CNN wie für das FCN ergeben. Es konnten hier kleine Verbesserungen im Nachkommabereich erbracht werden. Diese geringfügigen Unterschiede können aber vernachlässigt werden, da sie sich im Durchschnitt unter 1\% befinden. Somit zeigte sich bei dieser Optimierung keine Verbesserung der Klassifikationsergebnisse durch den Genetischen Algorithmus, im Vergleich zur Zufallssuche.

\begin{table}[h]
\centering
\caption{Ergebnisse der Algorithmen auf dem Cifa10 Datensatz}
\label{tab:cnn}
\begin{tabular}{lllll} 
\toprule
Iterations & \multicolumn{2}{c}{50} & \multicolumn{2}{c}{250}  \\ 
\midrule
Genetischer Algorithmus         & 78,74\% & 78,12\%      & 79,34\% & 78,49\%                \\
           & 78,18\% & 77,95\%      & 78,66\% & 78,49\%                \\
Zufallssuche         & 79,51\% & 79,36\%      & 82,87\% & 81,58\%                \\
           & 79,09\% & 78,88\%      & 82,87\% & 81,54\%                \\
\bottomrule
\end{tabular}
\end{table}

\subsection{Ergebnis der Hyperparameter Optimierung mit verkleinertem Datensatz}
Die Evaluationen der Optimierung mit verkleinertem Datensatz wird für das Fully-Connected Netz und das Convolutional Neuronal Network durchgeführt. Diese sind in Tabelle \ref{tab:fully_small} und \ref{tab:cnn_small} zusehen. Die Optimierungen entsprechen exakt den gleichen Abläufen der zuvor durchgeführten Optimierungen nur mit dem Unterschied, dass für diese Optimierung der Datensatz um 90\% verkleinert wurde.

Bei der Optimierung der Hyperparameter des Fully-Connected Netzes (\ref{tab:fully_small}) gibt es Unterschiede bei den Ergebnissen des GA im Vergleich zur Zufallssuche. Dies ist vor allem in der Berechnung von 250 Iterationen bei der kleinen Modell Architektur zusehen. Hier ist die Klassifikationsgenauigkeit besser als ein Prozent. Des Weiteren konnte im Durchschnitt ein halbes Prozent bessere Ergebnisse mit dem Genetischen Algorithmus im Vergleich zu der Zufallssuche erreicht werden. Dennoch sind die Verbesserungen nur sehr gering.



\begin{table}[h]
\centering
\caption{Ergebnisse der Algorithmen auf dem verkleinerten Mnist Digits Datensatz für 50 Iterationen}
\label{tab:fully_small}
\begin{tabular}{lllll} 
\toprule
Größe des KNN & \multicolumn{2}{l}{klein} & \multicolumn{2}{l}{groß}  \\ 
\midrule
Genetischer Algorithmus              & 95,58\% & 94,47\%         & 95,92\% & 94,92\%         \\
              & 94,46\% & 94,45\%         & 94,86\% & 94,88\%         \\
Zufallssuche            & 94,42\% & 94,44\%         & 94,92\% & 95,33\%         \\
              & 94,28\% & 94,32\%         & 94,81\% & 94,82\%         \\
\bottomrule
\end{tabular}
\end{table}

\begin{table}[h]
\centering
\caption{Ergebnisse der Algorithmen auf dem verkleinerten Mnist Digits Datensatz für 250 Iterationen}
\label{tab:fully_small_250}
\begin{tabular}{lllll} 
\toprule
Größe des KNN & \multicolumn{2}{l}{klein} & \multicolumn{2}{l}{groß}  \\ 
\midrule
Genetischer Algorithmus            & 95,67\% & 94,82\%         & 95,83\% & 94,65\%         \\
              & 94,81\% & 94,80\%         & 94,59\% & 94,61\%         \\
Zufallssuche            & 94,85\% & 94,41\%         & 95,25\% & 94,43\%         \\
              & 94,22\% & 94,24\%         & 94,42\% & 94,38\%         \\
\bottomrule
\end{tabular}
\end{table}

Bei der Optimierung der Hyperparameter des Convolutional Neural Networks (\ref{tab:cnn_small}) gibt es Unterschiede bei den Ergebnissen des GA im Vergleich zur Zufallssuche. Bei der Optimierung mit Hilfe des GA konnte keine Verbesserung erbracht werden. Die Ergebnisse der Klassifikation sind sogar bis zu 4\% schlechter. 

\begin{table}
\centering
\caption{Ergebnisse der Algorithmen auf dem verkleinerten Cifar10 Datensatz}
\label{tab:cnn_small}
\begin{tabular}{lllll} 
\toprule
Iterations & \multicolumn{2}{c}{50} & \multicolumn{2}{c}{250}  \\ 
\midrule
GA         & 58,7\%  & 56,01\%      & 57,8\%     & 57,39\%            \\
           & 56,28\% & 55,97\%      & 54,78\%     & 56,02\%            \\
ZS         & 61,6\%  & 59,58\%      & 60,7\%  & 62,59\%        \\
           & 59,65\% & 59,17\%      & 60,73\% & 60,87\%        \\
\bottomrule
\end{tabular}
\end{table}


\subsection{Ergebnis des Versuchs zur Modell-Architektur Optimierung} \label{versuch_modell}
Bei der Modell-Architektur Optimierung zeigte sich nach einigen Versuchen bei beiden Algorithmen eine Tendenz zu einer höheren Neuronenanzahl. Denn je mehr Neuronen desto mehr Informationen können vom KNN gelernt werden und führen dementsprechend zu besseren Ergebnissen. Somit wurde die Anzahl an Neuronen pro Schicht meist Richtung Maximum (maximale Anzahl Neuronen pro Schicht) optimiert. Um dies zu verhindern müsste eine spezielle Fitnessfunktion geschrieben werden, in welcher die gesamte Anzahl von Neuronen als negativen Einfluss mit eingebracht wird. Da die Optimierung der Modell-Architektur nicht zum eigentlichen Teil der Arbeit gehört, wurde nach einigen Test die Versuche zur Modell-Architektur Optimierung eingestellt. Ideen zur Weiterentwicklung der Modell-Architektur Optimierung werden im Ausblick \ref{ausblick} angeführt.

\newpage

\section{Ergebnis Visualisierung - Dichtediagramm} \label{ssec:Visualisierung}
Das Dichte-Diagramm dient als zusätzliche Auswertung des Genetischen Algorithmus. Es kann nur auf die Ergebnisse des Genetischen Algorithmus angewendet werden, da dieser genügend Individuen liefert die einen hohen Fitnesswert besitzen. Um eine Aussage zu den einzelnen Hyperparameter zu treffen, werden die Hyperparameter der letzte Generation in einem Diagramm dargestellt. Im Dichte-Diagramm werden die Hyperparameter einzelner Individuen über die Fitness aufgetragen. Hierfür werden schlechte Individuen, die einen Fitnesswert schlechter als 20\% des besten Individuums haben, heraus gefiltert. Somit werden nur Individuen mit hoher Qualität dargestellt. Durch das Auftragen der Hyperparameter, bezogen auf ihre Fitness, wird durch die visuelle Auswertung klar, welche Werte sich für einen speziellen Hyperparameter eignen. Dieses Dichte-Diagramm hat eine begrenzte Aussagekraft, da es sich um einen mehrdimensionalen Suchraum handelt, in dem sich alle Hyperparameter gegeneinander beeinflussen. Dies kann in den Dichte-Diagrammen nicht berücksichtigt werden. Trotzdem wird durch das Dichte-Diagramm ein Rahmen für jeden Hyperparameter sichtbar. Im nachfolgenden Kapitel werden die Diagramm für jeden optimierten Hyperparameter aufgezeigt und besprochen. Diese Diagramme werden aus den Ergebnissen der Hyperparameter Optimierung mit dem Fully-Connected Netz mit 250 Iterationen mit ganzem Datensatz erstellt. Es werden Diagramme zu allen Optimierungsvorgängen erstellt. Diese werden aber aus Platzgründen hier nicht aufgezeigt und sind im Anhang zu finden.
 
\newpage

\subsection{Diagramm Lernrate}
In Abbildung \ref{fig:hp_learningrate} ist die Lernrate in Verbindung mit der Klassifizierungsgenauigkeit aufgetragen. Es zeigte sich eine Lernrate im Bereich von 0,005 bis 0,1 als zielführend. Zudem ist die Lernrate im Bereich von 0,005 bis 0,05 bei den Individuuen der letzen Generation häufig vertreten, wodurch dies eine hohe Garantie für eine hochwertiges Individuum widerspiegelt. Da es aber einige Individuen eine Lernraten von 0.1 besitzen, die eine hohe Fitness aufweisen ist es nicht möglich einen besten Wert für die Lernrate festzulegen.

\begin{figure}[H]
  \centering  
  \includegraphics[scale=0.9]{img/hp_learningrate.pdf}
  \caption{Dichte-Diagramm der Lernrate in Verbindung mit der Klassifizierungsgenauigkeit(acc)}
  \label{fig:hp_learningrate}
\end{figure}

\subsection{Diagramm Dropout}
In Abbildung \ref{fig:hp_Dropout} ist der Dropout in Verbindung mit der Klassifizierungsgenauigkeit aufgetragen. Es zeigte sich ein Dropout im Bereich von 0,1 bis 0,4 als zielführend. Zudem ist der Dropout im Bereich von 0,2 in den Individuen der letzten Generation häufig vertreten, wodurch dies eine hohe Garantie für eine hochwertiges Individuum widerspiegelt. Im Diagramm ist auch gut zusehen, dass eine Dropoutrate über 0,5 sich kontraproduktiv für die Klassifzierungsgenauigkeit auswirkt. Dies ist über die geringe Klassifzierungsgenauigkeit(acc) für den Dropout über 0,5 sichtbar.

\begin{figure}[h]
  \centering  
  \includegraphics[scale=0.9]{img/hp_dropout.pdf}
  \caption{Dichte-Diagramm des Dropouts in Verbindung mit der Klassifizierungsgenauigkeit(acc)}
  \label{fig:hp_Dropout}
\end{figure}

\newpage

\subsection{Diagramm Batchsize}
In Abbildung \ref{fig:hp_Batchsize} ist die Batchsize in Verbindung mit der Klassifizierungsgenauigkeit aufgetragen. Es zeigte sich eine Batchsize im Bereich von 30 bis 50 als nützlich. Zudem ist die Batchsize im Bereich von 40 in den Individuen der letzten Generation häufig vertreten, wodurch dies eine hohe Garantie für ein hochwertiges Individuum widerspiegelt.

\begin{figure}[H]
  \centering  
  \includegraphics[scale=0.9]{img/hp_batchsize.pdf}
  \caption{Dichte-Diagramm der Batchsize in Verbindung mit der Klassifizierungsgenauigkeit(acc)}
  \label{fig:hp_Batchsize}
\end{figure}

\newpage

\subsection{Diagramm Epochen}
In Abbildung \ref{fig:hp_Epochen} ist die Epochenanzahl in Verbindung mit der Klassifizierungsgenauigkeit aufgetragen. Es zeigte sich, dass eine Epochenanzahl im Bereich von 20 bis 80 als zielführend. Zudem sind die Epochen im Bereich von 40 in den Individuen der letzten Generation häufig vertreten, wodurch dies eine hohe Garantie für eine hochwertiges Individuum widerspiegelt. In diesem Diagramm ist gut zusehen das ein Individuum mit höherer Epochenanzahl meist gute Fitnesswerte liefert, wobei eine geringere Epochenanzahl ab einem bestimmten Wert, hier 20, eine geringere Fitness aufweist.

\begin{figure}[H]
  \centering  
  \includegraphics[scale=0.9]{img/hp_epoch.pdf}
  \caption{Dichte-Diagramm die Epochenanzahl in Verbindung mit der Klassifizierungsgenauigkeit(acc)}
  \label{fig:hp_Epochen}
\end{figure}

\newpage

\subsection{Diagramm Optimierer}
In Abbildung \ref{fig:hp_Optimierer} sind die Optimierer in Verbindung mit der Klassifizierungsgenauigkeit aufgetragen. Es zeigte sich, dass der Optimierer 1 = SGD und 3 = Adagrad am häufigsten mit gutem Fintesswert auftreten. Es ist gut zusehen, das einzelne Optimierer wie 2 = RMSporp oder 7 = ftrl gar nicht auftreten, dementsprechend schneiden diese Optimierer schlecht ab. Somit erhalten wir durch das Dichte-Diagramm, dass SGD und Adagrad die favorisierten Optimierer sind.

\begin{figure}[H]
  \centering  
  \includegraphics[scale=0.9]{img/hp_optimizer.pdf}
  \caption{Dichte-Diagramm der Optimierer in Verbindung mit der Klassifizierungsgenauigkeit(acc)}
  \label{fig:hp_Optimierer}
\end{figure}
Jede ganze Zahl des Optimizer steht für einen eigenen Optimierer. Es gilt: 0=adam, 1=SGD, 2=RMSprop, 3=Adagrad, 4=adadelta, 5=adammax, 6=nadam, 7=ftrl.

\newpage

\section{Diskussion der Ergebnisse}
Bei den Optimierungen der Hyperparameter haben sich nicht die erwarteten Ergebnisse gezeigt. Bei der Optimierung der Hyperparameter der Fully Connected Netze zeigten sich nur sehr geringe Unterschiede in den Ergebnissen. Diese Unterschiede liegen im Nachkommabereich und können vernachlässigt werden. Somit konnte bei der Optimierung des Fully Connected keiner der Algorithmen sich als besser herausstellen. Bei der Hyperparameter Optimierung ergaben sich ähnliche Ergebnisse. Bei 250 Iterationen war die Zufallssuche sogar bis zu 4\% besser als der GA. Bei der Hyperparameteroptimierung mit kleinem Datensatz waren eine Verbesserung von bis zu 1\% beim Fully Conntected Netz mit Hilfe des GA möglich, dennoch ist diese Verbesserung irrelevant klein. Somit konnten in allen drei Optimierungsversuchen keine relevante Verbesserung der Klassifikationsgenauigkeit mit Hilfe des Genetischen Algorithmus gegenüber der Zufallssuche gezeigt werden. Dabei äußerte sich eine Änderung der Berechnungsdauer sowie der Modell-Architektur nicht in einer Änderung der Endergebnisse. Die geringe Verbesserung liegt nicht daran, dass der GA ungeeignet ist. Denn der GA konnte im Vergleich zur Zufallssuche häufiger Individuen finden, welche einen guten Fitnesswert besitzen. Dennoch gab es kein Individuum das verbesserte Ergebnisse lieferte. Somit konnte die Zufallssuche mit ihren wenigen gefundenen Individuen die gleiche Ergebnisse erreichen wie der GA. Dies kann durch einen geringen Einfluss der Hyperparameter auf das Endergebnis begründet werden. Kleine Veränderungen der Hyperparameter haben keine messbaren Änderungen der Endergebnisse ergeben. Solange die Hyperparameter in einem bestimmten Rahmen liegen, werden gute Klassifizierungsergebnisse erzielt. Dieser Rahmen für die Hyperparameter kann mit Hilfe des GA sichtbar gemacht werden. Durch die Population im GA kann gesagt werden in welchem Rahmen sich die Hyperparameter befinden müssen um gute Ergebnisse zu liefern. Diese Information konnte aus der Zufallssuche nicht entnommen werden, da die meisten Individuen der Zufallssuche keinen guten Fitnesswert besitzen. Dieser Rahmen zeigt sich in der Auswertung über die Visualisierung \ref{ssec:Visualisierung}, somit erreicht der Genetische Algorithmus gegenüber der Zufallssuche keine großen Verbesserungen. Doch durch seine Zusatzinformationen kann ein Mehrwert mit gleicher Berechnungszeit gewonnen werden.

\newpage

\section{Zusammenfassung}
In diesem Kapitel wurde auf die Ergebnisse und deren Auswertung eingegangen. Dazu wurden zuerst die Benchmarks des Trainingsvorgangs erklärt. Es zeigte sich, dass große KNNs deutlich schneller auf der GPU trainieren. Hingegen trainieren kleine KNNs auf der CPU deutlich schneller. Anschließend werden die Ergebnisse des Evaluierungsteils besprochen. Hier zeigten sich nicht die erwarteten Ergebnisse. Der Genetische Algorithmus konnte nur in wenigen Fällen eine Verbesserung der Klassifkationsergebnisse, über das Optimieren der Hyperparameter, erreichen. Zudem war die Verbesserungen vernachlässigbar klein. Denn die Zufallssuche konnte wenige Individuen mit guten Hyperparametern finden. Diese Individuen waren ausreichend gut, dass eine Optimierung mit Hilfe des GA keine Verbesserung gegenüber diesen Individuen erbrachte. Dennoch hat der Genetische Algorithmus einen Vorteil. Er kann über seine letzte Population eine zusätzliche Aussage treffen. Diese beinhaltet in welchen Bereichen sich die einzelnen Hyperparameter befinden, die gute Ergebnisse liefern. Diese Bereiche wurden mithilfe des Dichte-Diagramms sichtbar gemacht.

\section{Zusammenfassung und Ausblick}
\subsection{Einleitung}

\subsection{Zusammenfassung}

\subsection{Bedeutung der Arbeit}

\subsection{Ausblick}



%Literaturverzeichnis
\bibliographystyle{unsrt}
\bibliography{Literatur}

\end{document}
