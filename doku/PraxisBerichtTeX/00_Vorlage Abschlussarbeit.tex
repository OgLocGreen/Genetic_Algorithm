%% LaTeX template for FZI designed Document
%% by Matthias Huber and Florian Kuhnt
%%
%% version 0.4
%%
%% Problems, bugs and comments to
%% huber@fzi.de

\documentclass[parskip=true]{scrartcl}


% Toggle the following two lines to switch between english and german layout
%\usepackage[english]{babel} % English
\usepackage[ngerman]{babel} % Neue deutsche Rechtschreibung und Silbentrennung.

%\usepackage{templates/theme}

\usepackage{lipsum}
\usepackage{todonotes}	
\usepackage[nolist,nohyperlinks]{acronym}

\newboolean{titel} 
\setboolean{titel}{true}
\newboolean{inhalt} 
\setboolean{inhalt}{true}
\newboolean{headfoot} 
\setboolean{headfoot}{true}

% Dokumentangaben
\newcommand{\dokumenttitel}{[GA]}
\newcommand{\untertitel}{[Untertitel]}
\newcommand{\version}{0.1}
\newcommand{\veroeffentlichung}{1.1.1970}
\newcommand{\autor}{[Chrisitian Heinzmann]}
\newcommand{\validfrom}{1.1.1970}
\newcommand{\state}{-vertraulich-}

%Titelseite?
%\setboolean{titel}{false} 
%Inhaltsverzeichnis?
%\setboolean{inhalt}{false}
%FZI Kopf- und Fußzeile?
%\setboolean{headfoot}{false}

%Inhaltsverzeichnis?
%\setboolean{inhalt}{false}
%FZI Kopf- und Fußzeile?
%\setboolean{headfoot}{false}

\begin{document}

%Seiten ohne Kopf- und Fußzeile sowie Seitenzahl
\pagestyle{empty}

\begin{center}
\begin{tabular}{p{\textwidth}}


\begin{center}
\includegraphics[scale=0.4]{img/Hska_logo.png}
\end{center}


\\

\begin{center}
\LARGE{\textsc{
Genetische Algorithmen zur Optimierung von Hyperparametern eines k"unstlichen neuronalen Netzes\\
}}
\end{center}

\\


\begin{center}
\large{Fakult"at f"ur Maschinenbau und Mechatronik \\
der Hochschule Karlsruhe \\
Technik und Wirtschaft \\}
\end{center}

\\

\begin{center}
\textbf{\Large{Bachelorarbeit}}
\end{center}


\begin{center}
vom 01.03.2018 bis zum 31.08.2018 \\
vorgelegt von
\end{center}

\begin{center}
\large{\textbf{Christian Heinzmann}} \\
\small{geboren am 18.02.1995 in Heilbronn} \\
\small{Martrikelnummer: 52550}
\end{center}

\begin{center}
\large{Winter Semester 2019}
\end{center}

\\

\\

\begin{center}
\begin{tabular}{lll}
\textbf{Professor} & & Prof. Dr.-Ing. habil. Burghart\\
\textbf{Co-Professor} & & Prof. Dr.-Ing. Olawsky\\
\textbf{Betreuer FZI:} & & M. Sc. Kohout\\
\end{tabular}
\end{center}

\end{tabular}
\end{center}

\addsec{Eigenständigkeitserklärung und Hinweis auf verwendete
Hilfsmittel}
\label{erklaerung}

Eigenständigkeitserklärung und Hinweis auf verwendete
Hilfsmittel.
Hiermit bestätige ich, dass ich den vorliegenden Praxissemesterbericht selbständig
verfasst und keine anderen als die angegebenen Hilfsmittel benutzt habe. Die Stellen
des Berichts, die dem Wortlaut oder dem Sinn nach anderen Werken entnommen sind,
wurden unter Angabe der Quelle kenntlich gemacht.\\
\\[1.5cm]
Datum:	\hrulefill\enspace Unterschrift: \hrulefill
\\[3.5cm]

\addsec{Danksagungen}
\label{danksagungen}
An dieser Stelle möchte ich mich zuerst bei dem Forschungszentrum Informatik bedanken, durch die ich die
Möglichkeit bekommen habe, mich in einem innovativen Forschungszentrum auf meiner persönlichen und fachlichen Ebene weiter zu entwickeln.
Außerdem bedanke ich mich bei sämtlichen Kollegen der Abteilung Embedded Systems and Sensors Engineering, die
mich jederzeit hilfsbereit und kompetent unterstützt haben.
Mein besonderer Dank gilt Herrn Lukas Kohout, welcher sich
als Betreuer immer die nötige Zeit nahm, Sachverhalte zu erklären und meine Fragen zu beantworten.




\begin{figure}
\addsec{Ausschreibung}
\label{sec:Kurzfasssung}
\includegraphics[height=\dimexpr\textheight-0\baselineskip\relax,page=1]{img/Ausschreibung}
\label{fig:Praktikumsbeschreibung}
%\includepdf[pages=-]{img/Ausschreibung_Handlungspraediktion}


\end{figure}



% Beendet eine Seite und erzwingt auf den nachfolgenden Seiten die Ausgabe aller Gleitobjekte (z.B. Abbildungen), die bislang definiert, aber noch nicht ausgegeben wurden. Dieser Befehl fügt, falls nötig, eine leere Seite ein, sodaß die nächste Seite nach den Gleitobjekten eine ungerade Seitennummer hat. 
\cleardoubleoddpage

% pagestyle für gesamtes Dokument aktivieren
%\pagestyle{fancy}

%Inhaltsverzeichnis
\tableofcontents

%Verzeichnis aller Bilder
\listoffigures

%Verzeichnis aller Abkürzungen
\addsec*{Abkürzungsverzeichnis}
\begin{acronym}[Bash]
 \acro{KNN}{Künstliches Neuronales Netz}
 \acro{Conv}{Convolutional}
 \acro{ESS}{Embedded Systems and Sensors Engineering}
 \acro{FZI}{Forschungszentrum Informatik}
 \acro{CNN}{Convolutional Neuronal Network}
 \acro{GA}{Genetic Algorithmen}
\end{acronym}



\section{Motivation}
\label{sec:Motivation}
In den letzten Jahren zeigte sich ein enormer Zuwachs an Publikationen und Forschungen zum Thema "Maschinelles Lernen". Die künstlichen neuronalen Netze dominieren dieses Feld, aber ihr Training und ihr Erfolg hängt noch immer von empirisch ausgesuchten Anfangsparametern, den sogenannten Hyperparametern, ab. Zu diesen Hyperparametern können Modell-Architektur, Verlustfunktion, Optimierung Algorithmen, Batchsize, Dropout und Lernrate gezählt werden. Momentan werden diese Hyperparameter meist nach Ermessen des Entwicklers ausgewählt, denn es gibt keine formalen Regeln. Dieses Auswählen und Testen der Hyperparameter beansprucht sehr viel Zeit und Mühe, da die anfängliche Auswahl der Parameter meist suboptimal ist und diese noch einige Male angepasst und getestet werden müssen. Dieses Optimieren der Hyperparameter sollte jedoch nicht zu den Aufgabe des Entwicklers gehören und sollte von Maschinen übernommen werden, welche in Bereich der Optimierung wesentlich effizienter arbeiten. \cite[p.~337]{francois2017deep}

\section{Aufgabenstellung}
\label{sec:Aufgabenstellung}
Ziel dieser Arbeit ist es die Optimierung von Hyperparametern für künstliche neuronale Netze zu automatisieren. Diese Optimierung soll Mithilfe des Genetischen Algorithmus erfolgen. Dafür muss als erstes ein automatisierter Trainings- und Auswerte-Prozess der künstlichen neuronalen Netze erstellt werden. Anschließend soll ein Konzept zur Optimierung von Hyperparametern mit Hilfe des Genetischen Algorithmus entwickelt und implementiert werde, wozu der automatische Trainings- und Auswerte-Prozess verwendet werden soll. Die Ergebnisse des Genetischen Algorithmus sollen anschließend mit den Ergebnissen des State-of-the-Art Algorithmus zur Optimierung von Hyperparametern gegenübergestellt und ausgewertet werden. Darüber hinaus soll eine Optimierung der Hyperparameter unter Betrachtung geringer Datenmengen durchgeführt werden. 

\section{Aufbau der Arbeit}
\label{sec:Aufbau der Arbeit}
Im zweiten Kapitel wird auf die Grundlagen der künstlichen neuronalen Netze eingegangen. Anschließend werden die zu optimierenden Hyperparameter definiert. Nachfolgend werden die Optimierungsalgortihmen erläutert, dazu gehören der Genetische Algorithmus mit seinen 5 Schritten und die Rastersuche bzw. die Zufallssuche.

Im dritten Kapitel wird auf den Stand der Technik eingegangen. Dazu werden die neusten Forschungsergebnisse erläutert. Zu diesen gehört das Populations Basierte Training(PBT) von Google und die NeuroEvolution of Augmenting Topologies(NEAT). Zudem werden Anwendungsbeispiele des Genetischen Algorithmus, welche in der Forschung als auch in der Industrie verwendetet werden aufgezeigt. 

Im vierten Kapitel wird auf das Konzept zur Optimierung der Hyperparameter eingegangen. Hierfür wird ein Konzept zum Genetischen Algorithmus zur Optimierung der Hyperparameter eines künstlichen neuronalen Netzes vorgestellt. Anschließend wird auf das Konzept zur Evaluierung und Auswertung mit Hilfe des Dichte-Diagramms eingegangen. 

Im fünften Kapitel wird der Systemaufbau erklärt. Es wird auf die verwendete Hardware eingegangen. Anschließend wird die exakte Implementierung des Genetischen Algorithmus, der Zufallssuche und der Evaluation sowie die Auswertung beschrieben.

Im sechsten Kapitel werden die Ergebnisse der Optimierung dargestellt und besprochen. Zudem wird die visuelle Auswertung zu jedem Hyperparameter aufgezeigt und erläutert. 

Im siebten und letzten Kapitel werden alle vorherigen Kapitel als Fazit zusammengefasst. Der Ausblick zum Thema bildet den Abschluss der Arbeit. 





\section{Grundlagen}
\label{sec:Grundlagen}


\subsection{Optimierungsgrundlagen}
Angenommen es soll ein Neuronales Netz mit k Layern und jNeuronen zur Klassifizierung von einfachen Handgeschrien-Zahlen erstellt werden. Der Entwickler entscheidet sich für ein 3 Layern Netz mit jeweils 3 Neuronen. Nach dem Training hat es die Genauigkeit von 85 Prozent. Ist dies Akzeptabel? Kann man sagen das für k 3 bzw. j 3 die optimale Lösung ist? Um dies zu beurteilen müssen viele Experimente durchgeführt werden. Die Frage ist, wie man den die Besten werte für k und j findet, der die Klassifizierung maximiert. Dies wird als Hyperparameter Optimierung bezeichnet. 

\subsection{Genetische Algorithmen}
Genetische Algorithmen sind heuristische Suchansätzem, die auf einer breitenbasis von Optimierungsproblemen angewendet werden können. Diese Flexibiliät macht sie für die Praxis für viele sehr attraktiv. Die Evulution ist Grundlage des Genetischen Aglorithmuses. durch die aktuelle Vielfalt und der Erfolg der Arten ist dies schon alleine ein guter Grund sich diesen Optimierungs Algortihmus näher anzuschauen. Denn sdiese Arten sind in der lage sich an ihre Umgebung anzupassen und sich zu zu komplexen Strukturen zuentwickelen, und somit das überleben in verschiedesten Umgebungen eröglichen. Hierbei ist die Paarung und Entwicklung von Nachkommen eine der Hauptprinipen des Erfolges der Evolution. In diesem Kapitel werden wir die Grundlage der GEnetischen Algorithmen näher anzuschauen. Beginnen wir mit der grundlage das es sich bei den Genetischen Algorithmen um einen Polulations ansatz handelt. Anschließend wird auf die wichtigesten genetischen Operatioren vorstellen darunter gehöhren, Selektion, Crossover und Muttation

Seite - 11 Genetic Algorithm Essentials


\iffalse
Algorithmus 1 zeigt den Pseudocode des grundlegenden genetischen Algorithmus, der Folgendes kann
dienen als Blaupause für viele verwandte Ansätze. Am Anfang eine Reihe von Lösungen,die als Population bezeichnet wird, wird initialisiert. Diese Initialisierung wird empfohlen.um zufällig den gesamten Lösungsraum abzudecken oder um Experten zu modellieren und einzubinden.
Wissen. Die Darstellung bestimmt den Initialisierungsprozess. Für BitfolgeDarstellungen ist eine zufällige Kombination von Nullen und Einsen sinnvoll, z.B.das anfängliche Zufallschromosom 1001001001001 als typische Bitfolge der Länge 10. Der Hauptgenerationskreislauf des Genetischen Algorithmus erzeugt neue Nachkommen.Kandidatenlösungen mit Crossover und Mutation, bis die Bevölkerung vollständig ist.

Algorithm 1 Basic Genetic Algorithm
1: initialize population
2: repeat
3: 		repeat
4:			fitness computation
5:			crossover
6:			mutation
7:			phenotype mapping
8:		until population complete
9:		selection of parental population
10: until termination condition

Seite - 11 Genetic Algorithm Essentials
\fi 

\subsubsection{Initzialisierung der Population}
Der Klassiche Genetische Algorithmus bassiert auf einer Reihe von Kandidatenlösungen. Dee größe der Populaton ist auch die Anzahl der Lösungen. Jede Lösung kann als einzelnes Induvidum gesehen werden und wird durch ein Chromsone representiert. Es gibt verschiedene möglichkeiten diese Gene dazustellen, wie z.B binär oder Dezimal.
Figure \ref{fig:chromosome} veranschaulicht ein Beispiel, wie eine Population aus vier Individuen(chromosomes)mit je einem Chromosom. Ein Chromso, besitzt wiederum  vier gene. Jedes dieser Gene ist durch eine binären zahl repräsentiert. 

\noindent%
\begin{figure}[H]
  \centering  
  \includegraphics[scale=0.3]{img/Chromsome-s134-PracticalComputerVion.png}
  \caption{Beispiel einer Polulation mit 4 induviduen (Chromsomen) welche vier binäre Gene besitzen \cite{rashid2017neuronale} }
  \label{fig:chromosome}
\end{figure}

Diese anfangs Poulation wird meißt Zufällig inizialisiert. Hier durch ist es möglich die größte abdeckung des Suchsraums zu gewähren. Dadurch besitzt die erste Generation eine sehr geringe Fittness dies verbessert sich aber im Lauf des Trainings. Durch Selection werde die nicht unnötigen/Contra-produktiven Individumen aussotiert. Bevor dies passiert muss aber erst die Bewertung durchgeführt werden.


\subsubsection{Bewertung aka Grade / Fittnesfunktion}
Nun besteht die erste Generation(Generation 0) aus einer Population mit völlig zufälligen Induviduen. Diese werden anhand einer für sine anwendung speizellen Fitnessfunktion bewerte. Dabei werden nicht einzelnen Gene bewertet sondern das ganze Genom/Chromoson/Idividum. Es wird also nicht berücksichtigt welches Gene sich positiv bzw. negativ auswirklen. Als Rückgabewert gibt die Fittnesfunktion uns einen Fittneswert, dabei steht ein höherer Fittnesswert stehts für eine höher Qualität an Individum. Da nun alle Individuen der Population bewertet wurden kann eine neue Generation erstellt werden.


\subsubsection{Weiterentwickeln aka Evolve}
Dazu werden aus zwei Elternpaaren ein neues Kind erstellt. Um die Elternpaare auszusuchen gibt es verschiedene Optionen.Da nun die Eltern fest stehen, wird per Crossover aus den beiden Elternpaaren oder aus dem Elternpool ein neues Kind generiert. Um bei den Genen eine höhere diversität zu gelangen werden die Kindergene noch mit einer Mutation versehen. Somit kann man einen höheren Suchraum(Abdeckungsgrad) abdecken. Nach dem eine neue Kind generation erstellt wurde wird der ganze vorgang so lange wiederhollt bis die geforderte Fintess ereicht wurde.

\paragraph{Auswahl der Elternpaare aka Select Parents}
Um eine convergenz richtung Optimalem Maximum bzw. Minmum zu schaffen werden die besten Elternteile der geraden bewertenen Generation ausgewählt.
Für die Auswahl gibt es verschiedene Ansätze, die bedeutesten werden genannt und erläutert.
\begin{itemize}
\item \textbf{Auswahl proportional zu Fittnes}, hierbei spielt die im vohrigen Schritt berechnete Fintess eine große Rolle. Die Eltern werden nach dieser Fittness proportional als Elternteil ausgewählt und zum Elternpool hinzugefügt.
\item \textbf{Best 50 prozent},heißt aus der oberen hälfte der alten Generation werden alle Induviduen dem Elternpool hinzugefügt.Aus welchen dann zufällig die einzelnen Elternteile ausgewählt werden. Es mssen natürlich nicht immer 50 prozent sein, es kann sich auch um einen anderen Prozentsatz handeln.
\item \textbf{Tunier Auswahl}, in diesem Verfahren werden k induviduen der Population ausgewählt. Diese k Induviuen Treten dann wie in einem Tunier gegeneinander an. Das Individum mit dem Besten Fittneswert ein Elternteil ausgewählt. Hierbei wird auf den Elternpool verzichtet und direkt ein Kind aus zwei gewinnern erstellt.
\item \textbf{Comma selction}, Genetic algorithm essentails s.17
\item \textbf{Pluss selection}, Genetic algorithm essentails s.17
\end{itemize}


Nun wurden die Eltern ausgewählt um nun aus den Elterngeneration eine neue Kindergeneration zu erstellen, dies wird im nächsten Schritt erklärt.

\paragraph{Paarung aka Breed / Variation}
Aus dem Elternpool werden nun Nachkommen(Kinder) geschaffen. Alleine durch die Paarung(Chrossover) von qualitativ hochwertigen individuen wird erwartet, das die Nachkommen eine bessere Qualität besitzen als die ihrer Eltern.Aber wenn man nur die eigenschaften der Eltern übernimmt gibt es keine Garantie, dass die Kinder eine höhere Qualität erreichen.Es kann sogar dazu kommen, das negative eigenschaften mit übernommen werden. Da dies natürlich nicht gewollt ist gibt es eine einfach Verbesserungs möglichkei. Die Muation, hier wird jedes Gen noch einmal mit einer zufälligen Muation versehen welches ähnliche aber andere Lösungen hervorbringt. Nun gehen wir noch einmal genauer auf Operation Chrossover und Muation ein.


\iffalse
Somit müsste sich die Finttnes der nächsten generation verbessern. Um dies zu ereichen werden die  Gene noch modifiziert, durch corssover oder mutation. Somit wird der suchraum noch einmal vergrößert aber nur in der nähe der für gut empfunden Individuem bzw. dieser Gene.
Um aus den Einzelnen elternpaaren neue Individuen zu generieren wird das Verfahren/algorithmus Crossover verwendet. Bei Crossover kann es nun auch verschiedene möglichkeiten geben.
\fi


\begin{itemize}
\item \textbf{Crossover}, hierfür werden die Chromosome der Kinder Individuen bestimmt. Dazu werden entweder 50 Prozent des ersten Elternteils und 50 Prozent des zweiten elternteils verwendet wie im Oberenteil von Abbildung \ref{fig:chromoson crossover} zu sehen ist. Es gibt auch den Ansatz das ganz zufällig die Gene ausgewählt werden und dem Kind vererbt werden wie im Unterenteil der Abbildung \ref{fig:chromoson crossover} zu sehen ist. 

\begin{figure}[H]
  \centering  
  \includegraphics[scale=0.5]{img/crossover.png}
  \caption{crossover anhand eines einfachen binären Chroms. Das erste zeigt eine 50/50 crossover. Das zweite zeigt eine Zufällige auswahl ders Gens.\cite{Rashid2017} }
  \label{fig:chromoson crossover}
\end{figure}

\item \textbf{Mutation},hierbei wird jedes gens des Indivium zufällig mit einer zufälligen mutation versehen. Durch diese Mutation wird eine neue Inforamtion/Lösung in die nachfolgende Generation übergeben. 

\end{itemize}

\begin{figure}[H]
  \centering  
  \includegraphics[scale=0.5]{img/mutation.png}
  \caption{Muation eines Genes um höhere vielfältigkeit zubekommen.\cite{Rashid2017} }
  \label{fig:chromoson mutation}
\end{figure}


\paragraph{Austausch/ Scheife / Exchange}
Die neue Generation aus Kindern tauscht nun die alte Generation aus. Nun folgen die gleichen Schritte: Grade, Slection, crossover und mutation. 
Diese Schleife wird so lange durchegführt bis die Populationsfittnes das zuvor festgelegte Maximum erreicht. Wenn dies geschied gibt es viele Lösungen welche alle sehr ähnlich sein sollten. Aus dieser kann dann das beste Individum ausgesucht werden und als beste Lösung eingesetz werden. 


\subsection{Verschiedene Arten von GA}
\subsubsection{Standart-GA}
\subsubsection{Steady-State-GA}
\newpage

\subsection{Künstliche Neuronale Netze}

\begin{figure}[htb]
  \centering  
  \includegraphics[scale=0.5]{img/S36_Buildyourown.png}
  \caption{Künstliches Neuronales Netz mit drei Schichten je drei Neuronen \cite{Rashid2017} }
  \label{fig:neural_network}
\end{figure}


Hier zu sehen ist ein Künstliches Neuronales Netz mit drei Schichten \ref{fig:neural_network}. Dies wurde dem natürlichen Vorbild der neuronalen Netze im Gehirn nach empfunden. Die Kreise nennt man Neuronen oder auch Perseptron, mehrere Neuronen zusammen ergeben eine Schicht oder auch Layer genannt. Die Verbindungen repräsentieren die Gewichte, über diese kann einem Netz verschiedene Zusammenhänge von Input und Output antrainiert bzw. angelernt werden. Zum Training werden viele Daten benötigt, aus welchen das Netz \glqq Lernt\grqq{}. Dafür ist es wichtig, viele aufbereitete Daten zu besitzen, denn diese Netze brauchen viele Trainingsiterationen, bis das gewünschte Ergebnis zustande kommt. Ein Neuron besteht aus Eingängen, Gewichten und einer Aktivierungfunktion sowie einem Ausgang. Die Vernetzung mehrerer Neuronenschichten lässt ein Neuronales Netz entstehen.

\newpage



\subsection{Aufbau eines Neurons}
\begin{figure}[htb]
  \centering  
  \includegraphics[scale=0.5]{img/S41_Buildyourown.png}
  \caption{Aufbau eines Neurons \cite{Rashid2017}}
  \label{fig:neuron}
  

\end{figure}
\subsubsection{Eingang aka Input}
Bei dem Input handelt es sich um einfache xxxFloatwert dieser wird mit den einzelnen Gewichten verrechnet. Ein Neuron hat meist mehrere Eingangsgrößen, welche alle zusammen mit den Gewichten aufsummiert werden. Diese Werte werden zufällig initialisiert und per Training verbessert, somit handelt es sich um einen angelernten Werte, welche durch die Backproagation(Fehlerrückführung) verbessert werden.

\subsubsection{Offset aka bias}
Auf dieses Aufsummiertes Ergebniss wird anschließend ein Bias gerechnet, dieser führt zu einem besseren Verhalten beim Trainieren. Bei diesen Werten handelt es sich um angelernte Werte, die per Backpropagation verbessert werden und die Flexibitlität der Netze erhöht.


\subsubsection{Aktivierungs Funktion}
Die Aktivierungsfunktion kann man sich als Schwellwert vorstellen, ab wann das Neuron den Input weiter gibt. Es gibt verschiedene Funktionen, um diesen Schwellwert zu definieren. Je nach Aufgabe des Neuronalen Netze werden andere Aktivierungsfunktionen verwendet. Bei Klassifizierungen werden heute meist ReLu-Layer oder ein Weakly-ReLu Layer benutzt, diese verhindern das Vanishing- bzw. Exploding- gradientproblem beim Trainieren.

\subsubsection{Ausgang aka Output}
Wenn der Schwellwert überschritten wird, wird am Output durchgeschaltet. Dieser Output kann entweder mit einer nen Schicht Neronen verbundne sein oder direkt als Ausgang gesehen werden. Über welchen man anhand von xxxVariabelenwerten/Kommawerten die 
Von Input nach Output nennt sich ein Single-Forward-Pass. Wie hier beschrieben wird, kann ein Netz verschieden viele Layer besitzen mit verschiedenen Anzahlen von Neuronen.

\subsection{Verlustfunktion aka lossfunktion}
Die Verlustfunktion stellt ein ausgesuchtes Maß der Diskrepanz zwischen den beobachteten und den vorhergesagten Daten dar. Sie bestimmt die Leistungsfähigkeit des neuronalen Netzes während des Trainings und der Ausführung. Ziel ist es, im laufenden Prozess der Modellanpassung, die Verlustfunktion zu minimieren.

\subsection{Optimierer alt Gradientenabstieg}
Um die Fehlerfunktion zu minimieren wird als Werkzeug der Gradienten Abstieg benutzt. Diese ist nur möglich da ein Künstliches Neuronales Netz aus verketteten differenzierbaren Gewichte der Neuronen(Tensoroperationen) aufgebaut ist, die es erlauben duch anwendung der Kettenregel die Gradientenfunktion zu finden, die den aktuellen Parametern des Datenstapels werte des Gradienten zuordnet. Es gibt auch hier verschiedene Ansätze von Optimierern, welche die genauen Regeln wie der Gradient der Verlustfunktion zu Aktualisierund der Parameter verwendet wird hier könnte Beispielweise den RMSProp-Optimierer, der die trägheit des Gradientenabstiegsverfahren berücksichtet. Seite 83 - Deep Learning chollet


\iffalse
 Im Grunde werden dabei die Gewichte so angepasst, dass ein besseres Ergebnis entsteht und dadurch die Fehlerfunktion verringert wird. Wie das Wort Backpropagation schon sagt, wird von hinten nach vorne verbessert. Es gibt verschiedene Variationen von Gradientenabstiegen, welche verschiedene Vor- und Nachteile haben. Bei dem Trainieren des Netzes wurde der Momentum-Optimizer, welcher aus einem Gradientenabstieg mit Momentum aufgebaut ist.
\fi

\subsection{Hyperparameter}
Als Hyperparameter werden, in Bezug auf KNN's, meist die Anfangsbedingungen bezeichnet. Somit handelt es sich um die Learnrate (eng. Learningrate), der Abdeckunggrad(eng. Dropout), die verlustfunktion oder auch der Optimizer. In selten fällen kann man die Modelachitektur auch als Hyperparameter bezeichen. Für diese Hyperparameter gelten keine universellen Werte, sondern müssen je nach Daten und Funktion(oder KNN), speziell angepasst und verändert werden. Deshalb gibt es nur einige Regeln und grobe abschätzungen in welchem grenzen sich diese Hyperparameter befinden. 

\subsection{Zusammenfassung}

\include{06_taetigkeitsbericht}

%\chapter{Ergebnisse}
\section{Ergebnisse der Actionserkennung}
\label{sec:ergebnisse}
Es wurde ein Datenset erstellt welches aus 80\% selbst aufgenommen Videos und 20\% Videos aus der Uni Clinc Heidelberg.


Es wurde ein Neuronales Netz zur Action Erkennung gefunden. Für dieses Netz wurden Python Skripte erstellt, welche die Trainingsdaten so vor verarbeiten, dass damit ein Netz trainiert werden konnte. Anschließend wurde das 3D-ConvNet mit 6 Klassen: Gloves on, cleaning, unpacking, Gloves off, disinfekt, others mit jeweils 40 Videos trainiert. Die Evaluation gab gute Ergebnisse aus, das Netz hat eine Pression von 90\% erreichen. Mit diesen Ergebnissen konnte man zeigen, dass es möglich ist diese Netzstruktur für Ego-Action-Recognition zu verwenden.


\section{Verbesserungsvorschläge und Zukunftsaussicht}
\label{sec:Verbesserungen}

\subsubsection{Erweiterung der Trainingsdaten}
Die Trainerdaten können noch erweitert werden, einmal mit Realen Daten und weitern Klassen.
Oder mit Data-Argumentation, in welcher man durch Drehen oder einbringen von Farben künstlich mehr Daten erstellt und somit eine bessere Erkennung zubekommen.

\subsubsection{Realtime-Anwendung}
Die Schnelligkeit muss verbessert werden und die API muss angepasst werden, um es als Realtimeanwendung umsetzen zukönnen.
Dies übersteigt aber die Zeit in meinem Praktikum und ist deswegen noch zu entwickeln.

\subsubsection{Optical Flow}
Der Optical Flow wurde umgesetzt, dennoch ist es möglich, diesen zu Verbessern, durch verschiedene Vorverarbeitungen z.b. Bildstabilisierungen. Dies ist nicht getestet worden, könnte aber zu Verbesserungen führen, um Hintergrundstörungen aus den einzelnen Images zu filtern.

\subsubsection{Dritter Steam mit Händen}
Des Weiteren könnte man das Netz mit einem dritten Steam erweitern, um so den Focus auf spezielle Details zu legen.

\begin{itemize}

\item \textbf{Segmentation der Hände:}
Mit der Segmentation der Hände wäre es möglich speziellen Focus auf die Hände zulegen oder auf Video abschnitte mit Händen.

\begin{figure}[!htb]
  \centering
  \includegraphics[scale=0.5]{img/hands.png}
  \caption{Segmentation Hands  \cite{Bambach_2015_ICCV}}
  \label{fig:Optical FLow}
\end{figure}

\item \textbf{Hand-Pose:}
Mit der Pose Estimation der Hände wäre eine Weiterentwicklung der Segmentation. Mit der Pose Estimation der Hände wäre eine Weiterentwicklung der Segmentation. Dennoch wurde dies meist mit RGB-D Images umgesetzt. RGB-D Images enthalten noch weiter Tiefeninformation welche die GoPro nicht mit liefert.

\begin{figure}[!htb]
  \centering
  \includegraphics[scale=0.25]{img/Hand-Pose.png}
  \caption{Hand-Pose with RGB-D-Images   \cite{Hand-Pose-paper}}
  \label{fig:Optical Flow}
\end{figure}

\end{itemize}


%Literaturverzeichnis
\bibliographystyle{unsrtdin}
\bibliography{Literatur}

\end{document}
