\section{Stand der Forschung und Technik}
\subsection{Forschung}
In der Forschung werden Genetische Algorithmen häufig zufinden. Gerade die großen Unternehmmen in der IT branche erforschen sehr viel.

\subsubsection{Deep Mind}
 So etwa auch Googles Deep Learning ableger Deep Mind. Welcher für viele bahnbrechende neuerungen im Themenfeld Künstliche Inteligenz bekannt geworden ist. Unteranderem haben sie AlphaGo, die erste Künstliche Intiligenz welche einen Menschen in GO besiegte oder Alpha Start welche einen der Besten Menschen im Computerspiel Starcraft2 besiegte. In all diesen Projekten wurde das von DeepMind entwickelte PBT welches auf den GA aufbaut angewendet.
 

\subsubsection{PBT} 
Google Deep Mind ist eine sehr große Forschungsabteilung. 
Sie verwenden Algorithem die dem GA Sehr ähnlich(weiter entwickelt) ist und zwar das Population bassierende Training (eng. Poulation Based Training short PBT). Sie benutzen PBT auch zum anpassen von Hyperparametern speziel für ihre Reinforcment Learning Models. Deep Minds Variante des GA ist sehr viel komplexer. Sie haben ein Online Learn verfahren in welchem sie die Hyperparameter während des Trainings anpassen können, dies ist durch einen einen Server auf dem die daten gespeichert sind möglich. Dieser gibt ihnen auch die möglichkeit Asynchron und Parallel zu arbeiten. Im Durschnitt kontne sie ihre Ergebnisse noch einmal um bis zu 5 Prozent verbesser. 

\subsection{Software Testing with Ga}
Die Softwarebewertung spielt eine entscheidende Rolle im Lebenszyklus eines Software-Produktionssystems. Die Erzeugung geeigneter Daten zum Testen des Verhaltens der Software ist Gegenstand vieler Forschungen im Software-Engineering.  In diesem Beitrag wird die Qualitätskontrolle mit Kriterien zur Abdeckung von Anwendungspfaden betrachtet und ein neues Verfahren auf der Grundlage eines genetischen Algorithmus zur Erzeugung optimaler Testdaten vorgeschlagen. 
\cite{Keshavarz}


\subsection{Travelling Salesman Problem}
Einer der bekanntesten anwendungen ist das Travelling Salesman Problem, in welchem die kürzeste Route für einen Postboten berechnet werden soll. Doch je mehr Briefe der Postbote austragen soll umso mehr Variablen gibt es, sprich es wird wesentlich schwerer für ein fest geschrieben (eng. Hardcoded) Algorithmus den kürzesten weg zu finden. Für den Ga ist dies kein Problem da mit der richtigen Fitnessfunktion eine einfacher Rückgabewert der Funktion zu bekommen ist. Dementsprechend kann die Route einfach optimiert werden.


\subsection{Nicht it anwendungen}
Genetische Algorithem werden nicht nur in der It oder Technik angewendet. Es gibt auch Forschungen zur berechnung von Temperatur verlaufs der Erde  \cite{Stanislawska1} mit Genetischen Algorithmen oder auch für abschätzungen von Wäreme flusses zwischen Athmosphäre und Meereies in Polarregeionen \cite{Stanislawska2} benutzt. 

Die NASA hat eine Weltraumantenne mit hilfe von Genetischen algorithmen entwickelt. Es können also auch Konstruktionen mit hilfe von Genetischen Algortihmen erschaffen werden.


\subsection{Generativ Design}
Heute gibt es in manchen Computer-aided design (CAD) schon implementierungen von Generativen Design Werkzeugen. In dennen über Iterationen neue mögliche Designs auf Basis der Genetischen Evolutio. Sie bauen nicht auf den Genetischen Algorithmen auf sind aber nahe verwante und sollten nicht unterschätzt werden. 
Mit ihnen ist es möglich Bionische Stukturen für addetive Vertigung zu designen. Und sie speziell auf die Anwendung anpassen. So kann aus einem einfachen Frästeil ein wesentlich Leichtes und material spaarenderes Model entwickelt werden.

\noindent%
\begin{figure}[H]
  \centering  
  \includegraphics[scale=0.3]{img/Additive.png}
  \caption{Additives Design über mehrer Iterationen}
  \label{fig:Ablauf_kurz}
\end{figure}





\subsection{GLEAM}
General Learning Evolutionary Algorthm and Method ist eine vom Kit entwickelte Methode um Aktionsketen zu berechnen. Dazu gehörtz zum beispiel das Aufeinandern abstimmen der Maschinen in einem Maschinenpark um so genannte totzeiten der Maschinen zu verringern also die gesamt auslastung zu erhöhen.

Mit GLEAM wurde auch versucht die Stuerung von 6-Achsigen Robotorarmen zu verbessern. Es konnte gezeigt werden das die Steuerung mit Gleam funktioniert, dies wurde aber leider nie der neue Industie Standart. Es wird immer noch mit der klasschischen xxxSteuerungxxx  gearbeitet. 


\subsection{Reainforcment learning with GA}
Reinforcment learning ist möglichweise einer der größten Anwendungsgebiete der GA. Hierbei werden Neuronale Netze nicht mit hilfe von Gradienstieg training, wie im Gundlagen Kapitel besprochen. Sondern mit hilfe von Genetischen Algorithmen. Dabei wird das Neuronale Netz nicht 




\subsection{Zusammenfassung}